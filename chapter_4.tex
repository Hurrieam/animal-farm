\chapter{第四章}

到了夏天快要结束的时候,动物庄园的伟大壮举就被广为传唱了,至少大半个英格兰都知道了这些事。鸽子在信息传播中起了至关重要的作用,每天,拿破仑和雪球都会命令好几群鸽子去向附近的农场里的动物讲述造反的经过,并且教他们唱《英格兰兽》。

这段时期,琼斯先生很是郁闷啊,他花了大把的时间在威灵登的红狮子酒吧里,只要有人愿意听,他就向他们讲述他的冤屈——竟然被一群畜生撵出了自己的家园。

别的庄主起初仅仅是对琼斯表示同情,但是并没有给予实质性的帮助。因为他们都在心里打起了小算盘,想要为自己捞点好处。幸而,临近的两个庄园主的关系一直都很差。一个是面积很大的老式庄园,叫作福科斯伍德,但是庄园主皮尔金顿先生对这个庄园不怎么上心,他这个乡绅经常干的事不是打理农场,而是根据季节的变换或者钓鱼或者打猎,所以整个庄园的牧场基本荒芜了,树篱根本无人修剪,放眼望去满目萧瑟。另一个平彻菲尔德庄园与之相比虽然面积小了点,但是要整洁得多,这个庄园的主人叫弗雷德里克,是个奸诈不讲理的人,就因为锱铢必较已经身陷好几个官司了。这两个农场主谁也看不惯谁,双方就连在有共同利益的事情上都很难达成协议。

不过他俩倒是做了一件相同的事,那就是阻止隔壁庄园动物造反的消息传给自己庄园的动物,因为这事确实让他俩吓得不轻。起初知道这个消息的时候,他俩对动物们自己管理庄园的想法不屑一顾,觉得这场闹剧最多两个星期就会过去了。他俩四处散布谣言,内容是曼纳庄园(他们不能忍受动物庄园这个名字,依旧坚持称为曼纳庄园)里的动物们快要饿死了,为了争抢食物不停地打斗。然而一段时间过去了,那个庄园里的动物们显然没有饿死,这两个农场主不得不继续编造其他的谎话,说那里的动物们染上了可怕的邪气,他们吃同类的肉,用烧得通红的马蹄铁作为武器互相拷打折磨,还共同占有他们的雌性动物。两个农场主言之凿凿,这些就是造反带来的恶果。

然而,这样的传言是不被任何生物所相信的。另一个完全相悖的消息却广为流传:动物们建立了自己的动物庄园,他们赶走了人类,他们完全靠自己打理一切事务。虽然出现了不同版本,内容也不尽相同,但是主要宗旨是相同的。那一整年里,全国范围掀起了一股股大小不一的造反风波:一向温顺的公牛,突然就撒了野;羊糟蹋完苜蓿,还撞坏了树篱;母牛踢翻了奶桶;猎马奔跑到栅栏边突然停住从而把骑手成功地甩了出去。更让人不可思议的是,《英格兰兽》不论是调子还是歌词都以惊人的速度被广为传颂。人类虽然假装对这首歌表示轻蔑,但是怎么也掩饰不了他们的愤怒。他们说,他们不明白自己庄园里的动物怎么能如此堕落到去唱这种下流的歌。所以,但凡被发现唱这个歌的动物都会遭到一顿鞭打。尽管挨打也不能阻止这首歌被传唱,树篱上乌鸦从嘴里冒出的是《英格兰兽》,榆树上蹲着的鸽子发出的也是这个调子,铁匠铺里有这个调子,教堂的钟鸣里还有这个调子。人们仿佛被这个调子包围了,每当耳边响起这个调子,人们就不禁战栗,仿佛听到了他们即将要面临的厄运的预言。

时间很快到了十月,这时候谷子已经收割并且堆放好了,甚至一些已经脱了粒。一天,原本应该在外面的一群鸽子突然返回来,神色慌张都飞进了庄园的院子。之后很快就看见琼斯带着他的伙计们,还有五六个隔壁福科斯伍德庄园和平彻菲尔德庄园的人,闯进了五闩大门,沿着通向庄园的大道朝着庄园快步走来,他们人人手拿武器,琼斯拿了一支猎枪,其他人拿着棍棒,显而易见,他们是来夺回庄园的。

动物们异常冷静,因为他们早就料到了会有这一天,早就作好了完全的准备。雪球曾经在农场主的屋里翻出过一本描写凯撒大帝征战的旧书,算是军事战略系列吧,所以这次的指挥理所应当地落在雪球的肩上,只见他沉着地做出一个个决定,不出两分钟,动物们全部各就各位。

第一轮的攻击随着雪球的一声令下就开始了,此时这伙人刚刚接近庄园的窝棚。先是所有的鸽子飞到了这群人的头上,三五十只一起盘旋,从半空中一起拉屎。人们忙着躲避空袭,不料树篱后面突然冲出了一大群鹅,狠狠地啄他们的小腿肚子。人们握着手里的棍棒毫不费力地赶走了这群鹅,其实这只是战斗的前奏,目的就是让人类晕头转向。紧接着雪球带领动物发起了又一次的攻击,他率领着本杰明、穆尔丽和所有的羊冲进人群里,横冲直撞,对这伙突袭者又顶又戳,本杰明也使出了拿手招——尥蹶子。尽管动物们竭尽全力,这些手握凶器,连脚上也穿着带钉子的鞋的人还是让他们不容小觑。突然,动物们都听到了雪球发出了一声代表撤退的叫声,动物们很听话地掉头退回到院子里。

这些人类看到退回院子的动物,很自负地欢呼雀跃,于是他们没有经过思考就决定乘胜追击。这就是雪球给他们下的套。只见他们刚进到院子,之前藏匿在牛棚里的三匹马、三头牛,还有所有的猪就突然从他们身后冒了出来,顺便切断了他们所有的退路。伴随着雪球的又一声进攻信号,激烈的战斗正式打响,雪球带头身体力行地扑向琼斯,琼斯条件反射地举枪就射,子弹擦着雪球的背部飞过去,虽不致命,但是依旧在雪球的背上留下很深的几道血口子,不幸的是一头羊被射中毙命了。说时迟,那时快,雪球又撞了过去,两百多斤的身体仅仅碰了一下琼斯的腿,琼斯就倒在了一个粪堆上,猎枪很配合地也飞了出去。最惊心动魄的场景是拳师制造的,他就像一匹没被阉割过的成年种马,后腿发力撑起了整个身子,两只钉着铁掌的前蹄横扫一切,第一下就准确无误地击倒了来自福科斯伍德庄园的一个马夫,是打在了脑袋上,这个人倒在了泥坑里就不再动弹了。一看到有同伴没了气息,这些人吓坏了,忙不迭地扔掉棒子,转身就跑。但是深深的恐惧让他们丧失了方向感,所有的动物齐上阵,追着他们围着院子跑。这些人在混乱中被顶撞,撕咬,踢踹,踩踏,凡是动物们能使出来的招式他们全尝遍了,就连猫也不例外地从房顶上蹿下来,抓着一个放牛人的肩膀,用锋利的爪子狠狠地刺进他的脖子,放牛人疼得大声喊叫,声音透着凄惨的气息。就在他们快要绝望的时候,他们突然看到大门那里并没有动物,他们即刻抓住了这一丝生还的机会,夺路冲出院子,迅速朝着大路的方向狂奔,一路上伴着鹅群的啄咬,驱逐,终于成功地消失在了路尽头。就这样,为时不过五分钟的偷袭就在人类的败逃结局下结束了。

还有一个人类留在了院子里,就是之前被拳师打倒在地的马夫。回到院子,拳师用蹄子扒拉一下那个人,想把他翻过来,脸朝上,但是很失败。

拳师得出一个结论:“他死了。这不是我本意,我忘了我戴着马蹄铁了。你们相信我不是故意的吗?”

“这不是感性的时候,同志!”伤口还在流血的雪球大声说道,“战斗就是如此,只有死人才是好人。”

“但是我从没想过会杀生,对人依然如此。”拳师反复地重申,眼里还含着泪水。

不知是谁叫了一声:“莫丽怎么不见了?”

莫丽的确不知所踪。顿时,一种恐惧感油然而生,动物们开始担心是不是人类设计伤害了她,或者已经把她抢走了。结果大家却在她的马棚里找到了她,那时她正把头钻到了马槽里的草料中。原来刚才枪一响,她就躲了回来。刚才那个倒在泥坑里的马夫并没有死,而是晕倒了,他苏醒的时候动物们正在找莫丽,他就趁机溜掉了。

动物们重新聚集在院子里,战斗胜利带来的喜悦和兴奋使他们久久不能平静,每一个动物都扯着嗓子诉说自己刚刚在战斗中的表现。当下,他们就举办了一场庆功会。升起了代表动物庄园的旗帜,反复颂唱《英格兰兽》,接着为那只在战斗中牺牲的羊举行了庄重的葬礼,并在他的墓前栽种了一棵山楂树。雪球在葬礼上作了简短的讲话,意思主要是,动物们应该随时为庄园贡献自己的生命,在需要的时候。

在动物们的强烈提议下,他们决定设立军功勋章,“一级动物英雄”的勋章和称号当即授予了雪球和拳师,那是一枚铜质奖章(是在农具室里翻出来的很旧的装饰),被授权可在星期天和节假日里佩戴。还有一枚“二级动物英雄”勋章及称号被追认给了那只死去的羊。

大家纷纷为这次战争的称呼出谋划策,最后一致通过定为“牛棚战役”,因为就是在那里发动了至关重要的袭击。他们还找到了琼斯落下的枪,并且在原庄主的房子里翻出了备用的子弹。于是这杆枪有了新的用途,它被放在旗杆脚下作为礼炮,在两个纪念日里鸣枪,一个是十月十二日的牛棚战役纪念日,一个是施洗约翰节起义纪念日。
