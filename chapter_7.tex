\chapter{第七章}

那个冬天极其寒冷。狂风暴雨刚停,冰雨夹着暴雪就降临了,接着就是天寒地冻。这样恶劣的天气一直持续到二月,这期间,动物们一直都没有松懈,一直都在尽心尽力地为风车的重建劳作着,他们知道外界一直都在注视着他们,如果不能按期完成风车的重建,一定会让那些人类耻笑的。

不怀好意的人类放话说他们不相信是雪球毁掉了风车,而使风车倒塌的真正原因是墙壁太薄了。但是动物们并不认同人类的断言。虽然嘴上说不认同,但是他们却把墙砌得足有三英尺厚,整整是以前厚度的两倍之多。可想而知,这样建下来就需要更多的石头,动物们就需要继续去采集石头。但是因为长时间下雪,采石场的地面全被雪覆盖了,什么事都不能干。后来,天气变得干燥了,但是依然寒冷,动物们才能出门干一些活,但是对他们来说是痛苦不堪的工作,动物们已经不像以前那么满怀希望了。饥饿和寒冷困扰着他们。只有拳师和苜蓿还没有气馁。声响器时不时地就会发表一番精彩的演说,内容无外乎是劳动的乐趣和劳动的神圣,但是真正让动物们产生动力的却是拳师的表现,他像往常一样的勤劳还有他经常说出来勉励自己的那句嘶鸣:“我要更加努力工作。”

食物从一月份开始就匮乏了。因为谷物饲料的需求量大,耗费得多,没办法维持以后的生活,动物们不得不考虑用土豆来填充一部分对谷物的需求。但是很不幸的,不久他们发现地窖上面没有盖上足够厚的土,导致很大一部分土豆冻了,并且冻得发黑,有的已经发软坏掉了,仅剩下一小部分可以食用。动物们毫无办法,只能吃谷糠和萝卜充饥,已经没有其他的食物能提供给他们,这样的日子持续了很长时间。看来挨饿的日子离他们越来越近了。

目前,关乎他们生死的唯一要做的事情就是对外界掩盖他们的真实现状。人类因为风车事故而重新振作了,他们到处散布谣言,各种版本的,内容离奇的,但全是和动物庄园有关的。对于动物庄园的现状,外面已经传疯了,说动物们已经面临饥荒和瘟疫,生死一线,并且还起了内讧,不断地自相残杀,严重到吞食幼仔才能勉强度日的程度。这样的传言让拿破仑不得不下决心利用温普尔,因为外界一旦知道了粮食的真实状况,那产生多么可怕的后果都不为怪,现下只能通过温普尔让外界对庄园里的情况产生相反的印象。之前,温普尔每周到庄园里来,动物们都是能躲便躲,绝不与之有接触。而如今,一小部分动物被有意识地挑选出来去接触他,这一小部分动物里绵羊居多,他们被命令待在距离温普尔先生不远的地方,并且让他们之间看似闲聊的谈话有意无意地传到他的耳朵里,谈话的内容也被设定为饲料很丰富或者饲料已经增加等等。此外,拿破仑还命令动物们把已经空空如也的堆在饲料棚里的那几个大箱子装上满满的沙子,用仅剩的饲料覆盖在沙子上,伪造成粮满仓。然后找了个合适的时机让温普尔在穿过饲料棚时看见,这个做法看来天衣无缝,温普尔信以为真,他出了庄园就向外界宣传,动物庄园里不可能发生过粮食短缺的现象。

即便骗过了外界,但问题依旧存在,并且越来越严重,维持到了一月底,就不得不想办法从别的地方搞到粮食。拿破仑在这些天里基本上没露过面,他整天待在庄主的院里,每一道门都有凶神恶煞的狗蹲守着。他出来的方式也是很正规的,六条狗组成队簇拥着他,不允许任何动物靠近,一旦靠近,狗们就龇牙狂吠以示警告。现在每个星期天早上,拿破仑也不轻易露面了,一般是通过另一头猪来传达他的命令,一般这个殊荣就落在了声响器头上。

又一个星期天早上,声响器传达了这样一个消息,所有的母鸡,下的蛋都要上缴。原因是拿破仑同温普尔签订了一个合同,合同规定每周由庄园提供四百枚鸡蛋。当然换回的钱要购买足够的谷物和粗粉,用于维持动物们的基本生活直至夏季的到来,那时,情况就会得以改善。

这个命令一传达完,母鸡们的抗议声就炸开了锅。即使之前给她们打过这样的“预防针”,会有一些牺牲,但是那时的她们根本不相信真的会有这一天的到来。这个时候,她们刚刚准备好一窝蛋用于开春孵化小鸡,她们认为现在拿走这些蛋,无疑为图财害命。为了让拿破仑收回这道命令,三只年轻的米诺卡小黑鸡带领母鸡们作出了反抗,方法是飞到椽子上产蛋,后果是鸡蛋全都落在了地上,摔碎了。面对自驱逐琼斯一役以来的第一次带有造反意味的举动,拿破仑毫不留情地作出了一道新指令,停止母鸡们的饲料供应,并且如果有接济发生,哪怕只是一粒玉米,也会对接济者处以死刑,执行者由狗负责担任。这样的对抗仅仅只持续了五天,母鸡们就败下阵了,她们乖乖地回到她们的窝里产蛋。九只母鸡在这次反抗活动中死去了,尸体被埋在了果园里,死亡原因被要求说成是感染球虫病。这次风波没让温普尔知道半点,一辆食品专用敞篷车每周准时出现在庄园,拉走规定好的鸡蛋。

雪球消失了,这一段时间动物们都没有再见过他。听说是躲到了附近的那两个庄园里的一家,或者在福科斯伍德庄园,或者在平彻菲尔德庄园。这段时间,动物庄园和周围几个庄园的关系有所缓和。碰巧庄园里有一堆木材,温普尔建议拿破仑把它们用作交易。因为木材是十年前整理完榉树林后留在那儿的,现在已完全风干,非常耐用,两个庄园的庄主皮尔金顿先生和弗雷德里克先生都有意向买下这批木材。拿破仑因此犹豫不定。他的态度一直不明朗,奇怪的是外界关于雪球的传言也在不断地变化,拿破仑有意跟弗雷德里克先生签约时,传言就说雪球躲在福科斯伍德庄园,而他改变主意与皮尔金顿先生洽谈时,传言马上就变成雪球一定在平彻菲尔德庄园。

又一个让大家炸开了锅的消息在初春时节突然传出来。雪球常常在夜半时分溜回庄园,更可恶的是他是回来搞破坏的。谷子少了,牛奶桶被打翻了,鸡蛋也被打破了,苗圃惨遭践踏,就连果树皮也被啃咬了。这一切在动物们看来,毫无疑问都是雪球干的,他们吓得躲在窝棚里夜不能寐。从此后,雪球成了一切搞破坏的代名词,但凡有一点小差错,那根本不用思考,第一反应肯定是雪球所为。如果有一扇玻璃窗子被打破了,或者有一个排水沟堵住了,马上就会有动物跳出来指认是雪球在半夜干的。当饲料棚的钥匙找不到了,动物们也无比坚信地认为是雪球把它扔进了井里,即便后来钥匙被在一袋面粉底下找到了,动物们还是深信依然是雪球干的。母牛们对外指责雪球半夜溜进牛棚,趁她们睡着了吸走了她们的奶。因为老鼠们异常猖獗,雪球被认定又多一个种族的同伙。

调查雪球的行动在万众瞩目中由拿破仑同志亲力亲为。几条狗组成的护卫队跟随他对庄园里的窝棚开始了一番仔细的巡查,其他的动物恭恭敬敬地保持着几步的距离跟随着他。走几步拿破仑就会俯下身子对着地面嗅一嗅,看是否有雪球留下的气味,他说他能够凭气味辨别出雪球留下的脚印。他保持着这个步调嗅完了整个庄园的窝棚,一个角落也不放过,大谷仓,牛棚,鸡舍,还有菜园无一例外都发现了雪球来过的痕迹。只见他的嘴拱到地上,狠狠吸上几口气,然后发出一种暗含惊讶的高调声音:“这里,雪球来过这里,我保证,我准确无误地嗅出了!”只要一听到“雪球”这个代名词,狗们立刻像条件反射似的张开血盆大口,冲着拿破仑闻过的地方狂吠不止,那声音让动物们都惊恐不已。

大家让这次的调查搞得失魂落魄,他们觉得雪球无处不在,就像一种看不见的势力,围绕在他们身边,随时能让他们遭受各种各样的威胁。晚上,声响器把神色慌张的大家紧急召集到一起,大家不免猜测是不是有重大的事情发生了。

果不其然,声响器神经质地跳来跳去,大声地叫道:“同志们,我们发现了一件极其恐怖的事情,雪球,那个叛徒现在正在平彻菲尔德庄园,他和弗雷德里克正在策划一场阴谋,他们要偷袭我们,妄图夺走我们的庄园。战争一开始,雪球就会给他们领路。更让我们不可思议的是我们低估雪球了,原本我们以为他造反是想要满足他的野心和狂妄自大,但是错了,那不是主要原因。同志们,你们知道真正的原因是什么吗?我来告诉你们,雪球从一开始就跟琼斯是狼狈为奸的!他是潜伏在我们队伍里的间谍。我们就是从刚刚找到的他留下来的没能及时带走的文件中论证这一点的。同志们,这完全能解释很多问题了,不是吗?牛棚战役中,你们不是亲眼见证了他是如何想让我们战败和毁灭的吗?值得庆幸的是,我们并没有让他的阴谋得逞。”

动物们听完声响器的话后全都震惊了。雪球这样的罪行与破坏风车相比,那就是不可饶恕的罪孽。他们面对声响器控诉的雪球的罪行,还是迟疑了好大一会儿。他们记得,貌似是记得的,在牛棚一战中,雪球是冲在最前面的,带领大家,并且给予大家莫大的鼓励和安慰,即使被琼斯的子弹伤到了脊背,他也没有停下战斗的脚步。动物们首先就很不理解这一点,他这么做怎么能是支持琼斯的呢?此刻就连拳师也困惑不已,他以前几乎不问为什么的。他躺了下去,把两个前腿压在身子底下,闭着眼睛开始费力地思考这个问题。

半晌,他说:“这个事情我不相信。我亲眼看见雪球在牛棚战役中英勇善战,不畏牺牲。而且,战争结束后,我们不是还授予了他‘一级战斗英雄’的勋章吗?”

“那是我们犯的错误,”声响器回答道,“我们现在不是弄明白了吗,同志们,他当时那么做的目的是想诱惑我们走向毁灭。他在他的文件中记录得清清楚楚,他没带走是他的不幸,但对我们来说是万幸。”

“但是他受了伤,并且流了血还在继续战斗。”拳师继续提出疑问。

“这很容易解释,那根本就是他早就安排好了的。”声响器叫嚷道,“琼斯是故意用子弹擦破雪球的一点皮的。你可以看看他留下的文件,你要是识字就可以看。他们之前商量好的就是让雪球在关键时刻发出撤退信号,这样就可以把庄园拱手送人了。他就差一点成功的时候,我们的领袖——英勇的拿破仑同志,及时出现了,我甚至可以说,要不是拿破仑先生,他们的阴谋就得逞了。你们仔细回忆一下,当时雪球在看见琼斯和他的伙计们冲进院子的一瞬间,转身就往回跑,你们中的好多也跟着他往回逃,恐慌迅速蔓延。就在那个紧要关头,我们就快沦陷的时刻,拿破仑同志挺身而出,他在大喊了一声‘消灭人类!’后立即冲上去死死地咬住了琼斯的一条腿,这样的场景你们忘了吗?同志们,这样的场景怎么能忘了呢?你们不会忘的,对吧?”声响器又使出他的拿手好戏,一边左右蹦来跳去,一边大声地叫喊。

动物们听着声响器绘声绘色地描述当时的那个场景,他们好像还真就想起了那样的画面,至少,他们还记得雪球确实在那一仗的紧要关头往回跑过。但是拳师还是觉得哪里不对劲,很不安心。

他思索了很久,还是开了口:“我还是不相信,至少不相信雪球从开始就是个叛徒。至于他后来的那些作为,我不否认。我认为他在牛棚战役的时候还是一个好同志的。”

对于拳师的话,声响器只是用非常缓慢而又极其坚定的话语算是给他的回应:“我们的领袖,拿破仑先生已经明确地——明确地,同志们——认定,雪球自始至终都是琼斯的奸细。嗯,是的,追溯到还没有谁想到要造反之前就是了。”

“哦,那就是另外一个事情了!”拳师回答,“只要是拿破仑同志说的,就一定不会是错的。”

“这样的态度才正确嘛,同志们!”声响器满意地叫道。即便如此,还是有动物发现他用他闪闪发亮的小眼睛狠狠地瞥了拳师一眼。他转过身的同时停顿了一下,用更重一点的语气补充了一句:“我好心提醒你们一下,睁大你们的双眼。我们有理由相信,雪球还有同伙潜伏在我们中间。”

在四天后的下午,动物们得到了拿破仑的命令全都聚集在了院子里。拿破仑等到所有的动物都到齐了,才佩戴着他的两枚勋章(因为他最近刚授予自己一枚“一级动物英雄”勋章和一枚“二级动物英雄”勋章)从屋里走出来,那九只高大凶猛的狗一直紧紧地围绕在他身边,发出让所有动物都不寒而栗的阵阵狂吠。大伙都默默地蜷缩在自己的位子上,空气中的气压很低,预示着一场暴风骤雨即将来临。

拿破仑站住后,脸色挂着严厉的表情,他扫视了一圈他的听众,然后发出了一声尖叫。狗们听到这声尖叫后立刻冲上前去,咬住了四只猪的耳朵就把他们往外拖。不管猪发出多么惊恐的哀嚎和求救,不顾猪的耳朵鲜血直流,狗们一路把这四只猪拖到了拿破仑脚下。尝到血腥味的狗,一时之间兴奋得要发了疯。更让每个动物吃惊的是,有三条狗竟然扑向了拳师。拳师当机立断伸出一只巨掌,把还在半空中的一只狗抓住后,按在地上。那条狗哀叫着告饶,另外两只狗吓得灰溜溜地退了回去。拳师用眼神征求拿破仑的意见,想知道如何处理这只狗,是踩死还是放掉。拿破仑看到这场面后,脸色陡然就变了,他厉声喝令拳师放掉狗。拳师抬起蹄子,那只狗哀嚎着负伤跑了回去。

吵闹声马上停止了。那四头猪浑身发抖,挤作一团,等待着对他们的审判,他们就是之前曾经抗议过拿破仑废除星期天早上碰头会的四只小猪,此刻他们脸上的每一道纹理仿佛都刻满了他们的罪状。拿破仑厉声要求他们自己坦白罪行。不等进一步的逼问,他们便开口一一交代了:自从雪球被驱逐后,他们之间一直保持着秘密的来往,最初的摧毁风车,是他们协助雪球完成的,并且还和他达成了一项协议,准备把动物庄园送给弗雷德里克先生。还有就是雪球一直都是琼斯的间谍,这是雪球曾经偷偷透露给他们的。他们刚一坦白完毕,狗们没给他们丝毫喘息的时间,就扑过去撕裂了他们的喉咙。完事之后,拿破仑用冰冷的声音问在场的其他动物,还有谁,还有什么要坦白的。

在鸡蛋事件上带头反抗的那三只母鸡走上前去,供认了雪球在她们梦里指示她们违抗拿破仑的命令。她们被执行了死刑。一只鹅也走上前坦白了他曾经在去年收割时私藏了6穗谷子,在深夜拿出偷吃了。随后一只绵羊也交代了雪球强迫她在饮水池中撒尿。后来又有两只羊坦白了他们曾经杀害了一位拿破仑的忠心追随者——一只老山羊,杀害的手段就是趁他患病咳嗽时,围着火堆追赶他,使他一直绕来绕去。这些交代完罪行的动物都被就地正法了。一个接一个的坦白,一个接一个的死刑,拿破仑的脚边很快就堆起来数不清的尸体,空气中到处都是浓郁的血腥味,这样的情形很久没出现了,自从琼斯走后。

一切坦白和杀戮结束后,剩下的动物们惶惶地互相挤着离开了,只有狗和猪还留在原地。动物们胆战心惊地想要分清楚,到底哪件事情带给他们的震惊更为巨大一些,是雪球和外人的勾结背叛,还是刚刚亲眼所见的对待背叛的残忍的惩罚。琼斯还在的时候,这样的屠宰经常发生,但是远不如这次给他们带来的震撼大,如今的情况糟糕极了,因为这是发生在自己同志之间的。琼斯离开之后,一直都没有发生过杀戮,就连一只小老鼠都不曾遭到过杀害。动物们漫无目的地走着,此时他们已经走到了小山包上,已经建好一半的风车矗立在那里。他们挤在一起趴了下来,像是要相互取暖。这些动物里有苜蓿、穆尔丽、本杰明、牛、羊还有一群鸡和鹅,唯独没有看见那只猫。其实在拿破仑宣布集合的时候,那只猫就不见了踪影。大家都闭嘴不语地卧着,只有拳师烦躁不安地走来走去,他的那条长长的黑色尾巴不断地抽打着他自己的肚皮。他实在难以接受,一丝低低的嘶鸣声泄露了他的心情。最后,他忍不住开了口:

“我还是难以理解,我怎么也想不到我们庄园里会发生这样的事情,造成这样的原因一定是我们犯了某些过失。依我看来,只有更加努力的工作才是解决这个问题的最好办法。所以,从今天起,我决定,以后我要提前整整一个小时起床。”

说完,他迈着沉重的步子朝着采石场那边走去。走到那儿,他一口气装了两车石头,直到把石头全都卸到了风车工地上,他才回马厩里睡觉。

其他的动物一直躺在小山包上,他们一直默默地围在苜蓿身边。在这里可以看到整个乡村的景色,绝大部分庄园也能收入眼里:那能一直延至大路的狭长的牧场,已经被耕犁过的地里长着茂盛碧绿的麦苗,草料地,灌木丛,饮水池,庄园里红色的屋顶上,烟囱里现在冒出了袅袅的炊烟。现在是春天的傍晚,万里无云,夕阳把它金色的余光涂上了茂密的草地和葱郁的树林,使它们闪着金子般的光芒。此时此景,刺激到了动物们,他们突然意识到了这是他们的家园,每一寸土地都是属于他们的,是他们的共有财产。这一刻呈现在动物们眼前的正是他们魂牵梦萦,一直向往的地方。苜蓿的双眼饱含着泪水注视着下面的山坡。给她一个用言语表达想法的机会的话,她一定会说,现在的状况已背离了以前的目标,几年前他们制定的为推翻人类而努力奋斗的目标不是这样的,老少校第一次鼓动他们造反,那晚所热情期盼可不是现在这样恐怖的杀戮的惨状。对于未来的唯一畅想就是能有一个这样的社会:摆脱饥饿,不受鞭子的折磨,动物们相互平等,在劳动中尽其所能,有能力的保护弱小的,就如当初老少校演讲的晚上,她用前腿保护那些失去了妈妈的小鸭子一样。现在的她无比困惑,从什么时候起他们不再敢表达自己的真实想法了。那些凶残的狗就在她面前咆哮,并且残忍地撕碎了交代了吓人罪行的同伴,她只能眼睁睁地瞅着却一点办法也没有,她的大脑里甚至连反抗的念头都没有。但是,她还是知道,这样的局势也要比原来琼斯在的时候强多了,再说了,防止人类卷土重来应该是他们眼下最为重要的事情。她就是这样,无论发生了什么样的事情,都对拿破仑同志忠心耿耿,接受他下达的任务,然后勤勤恳恳,努力完成。但是,这毕竟不是她和其他动物所曾殷切期盼的并为之努力工作的生活,他们积极努力地建造风车,被破坏了之后又加大力度重建,不畏生死地坦然面对琼斯的枪林弹雨,并不是想要过如今的这种生活。她只能把她的想法表达成这样,虽然她觉得还是不够准确。

苜蓿想了想,觉得什么都不能够准确表达出她现在心中的想法,只能换作唱歌了,她带头唱起了《英格兰兽》。围在她身边的动物们也跟着她开始唱,他们一遍又一遍地演唱,唱得缓慢而忧伤,但是配合得非常和谐,音调悦耳,他们在此之前从来没有这样演唱过这支歌。

当他们唱完第三遍,声响器就带着两条狗跑来阻止他们继续演唱,从他的脸上就可以看出要有重大的事情宣布。果不其然,他奉了拿破仑的特别旨意来通知他们,《英格兰兽》被废除了。从今以后,禁止再唱这首歌。

动物们被这个命令惊呆了。

穆尔丽第一个反应过来,随即她大声问道:“这是为什么?”

“因为不再需要了,”声响器冷冷地回答她,“《英格兰兽》这首歌是造反歌,我们造反已经成功了,今天下午处决完那群叛徒就象征着我们彻底地造反成功。我们已经把外部和内部的敌人都打垮了。《英格兰兽》所表达的是我们对未来社会的美好渴望和期盼,我们现在已经建立了这样的社会了,这首歌的目的达到后就没有价值了。”

动物们即便很惊恐,但还是不能接受,他们刚要抗议,绵羊们就大声咩咩地叫嚷开来:“四条腿好,两条腿坏”,这个老调子总是在关键的时候出现,这样持续了几分钟,关于这首歌要不要继续唱的争议也就这样不见了。

《英格兰兽》就这样被取缔了,随之由诗人小不点写的一首歌取而代之,它的开头是这样的:

动物庄园,动物庄园,

我坚决不能让你受伤!

在这之后的每个星期天早上,升旗之后演唱的就是这首歌。不知是什么原因,动物们就是觉得不论是曲调还是歌词,这首歌都赶不上原来的《英格兰兽》。
