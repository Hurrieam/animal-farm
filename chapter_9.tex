\chapter{第九章}

过了这么久,拳师的蹄子还是没有好,那个裂口依然在愈合中。风车的又一轮重建工作在庆祝活动结束的第二天就开始了。拳师更加卖命地干活,他一天都不休息,绝不闲着,而且为了荣誉,他忍着伤痛干活,都不让其他的动物察觉到。但是到了晚上,他偷偷地告诉苜蓿,他的蹄子疼痛得难以忍受了,苜蓿只好把草药嚼碎了敷在他的蹄子上。苜蓿和本杰明都劝他不要再这么拼命了,苜蓿告诉他:“一个马的肺是不能这样长期持续下去的。”但是这些劝告对他来说根本没用,他说能够在他退休的时候看到风车顺利地运转起来,是他现在唯一的一个尚未实现的真切的志向。

最近关于退休的讨论越来越多,当初在第一次制定动物庄园律法的时候,对退休的年龄给出了规定:猪和马是十二岁,牛十四岁,狗九岁,羊七岁,鸡和鹅五岁。并且律法里还包括了要发放充足的养老津贴,虽然直到现在也没有真正靠养老津贴生活的动物,据说养老津贴很丰厚,每匹马的是每天五磅谷子,到了冬天为每天十五磅干草,而且公告节假日还会额外发一根胡萝卜,或者有机会得到一个苹果,拳师在明年夏天就能正式退休了。可是现在,苹果园那边的本来作为退休休息地的牧场现在已经空出来作为大麦地了,有传言说会在大牧场的一角围起来建立一个退休动物的放牧场。

那个冬天跟去年一样难熬,同样寒冷,而且食物更加短缺。所有的动物都被削减了食物份额,这里面不包含猪和狗。声响器不论在什么情况下都能向动物们证明,他们的粮食并没有短缺,即使现在对粮食的供应上有了一些调整(他不承认这是“减少”,只说是“调整”),这也是暂时的。他说,在食物分配上过分教条平均是违反了兽主义的原则的。他发出尖细的嗓音拖出一连串的数字,有力地证明了现在和琼斯时代的差距,那是进步了一大截啊,现在他们拥有更多的燕麦,更多的干草和更多萝卜,而且现在的工作时间比以前短,饮用的水质更好了,动物们的寿命也长了,后代的存活率也升上去了,窝棚有了更多的干草,所以更加舒适了,就连跳蚤也明显减少了对他们的侵扰。他说了这么多的话,动物们都没有感到怀疑,琼斯时代是什么样子已经基本上从他们的记忆中抹去了。他们如今的生活其实常常食不果腹,饥寒交迫,眼睛一睁就开始干活,一直干到睡觉。但是他们乐意相信以前的日子比这更糟糕,更窘迫。最后声响器永远不会忘了指出一点,至关重要的一点,就是现在他们是自由的,是不受奴役的。

秋天的时候,四头母猪基本上同时产了崽,加起来一共是三十一个小猪崽,所以现在需要饲养的动物更多了。拿破仑是院子里唯一的种猪,小猪们清一色的都是小花斑,他们的身世就不难猜测了。公告已经出来了,动物们都知道再过一段时间,把砖头和木材买齐后,就专门给小猪们盖一间教室,位置在庄园主院子的花园里。目前拿破仑在院子的厨房里亲自教导他们,他们平时被勒令禁止同别的年幼的动物玩耍,活动区域基本是在花园里,主要就是健身。一条新的规定在这个时候出台了:当一头猪在路上与别的动物碰上了,那么其他的动物就必须站到路边,还有,猪们不论身份地位,都享有一项特权——在星期天能在尾巴上系饰带。

今年的收成不错,但是依然筹不到足够的资金。因为要用钱的地方很多,盖教室要花钱买砖头、石灰和沙子,此外,风车所需的机械设备也需要资金,现在就得慢慢积攒了。还有一些零碎的钱要花,比如只有庄主院子里才需要的灯油和蜡烛,还有拿破仑自己才能享用的糖(他说会发胖,所以严禁别的猪吃),除此以外,还有很多日常使用的易耗品,诸如工具、钉子、绳子、煤、铁丝、铁块和狗食饼干等等,这些东西看起来不起眼,一个也不值几个钱,但是仔细算起来也是不小的开支。已经卖掉了一个干草垛和一部分收回来的土豆,而且每周鸡蛋的供应数量从之前的四百枚涨到了六百枚,但是这样造成了这一年中,所孵出的小鸡数量没有办法使鸡的总数量保持原来的标准。动物们继去年十二月份减过一次口粮后,二月份又遭到了一次削减,为了省下买灯油的钱,窝棚里早就不允许点灯了。但是那些措施根本没怎么影响猪们,他们过得似乎还是挺滋润的,而且体重还在增加。二月末的一天下午,隔着院子就能闻到一股无比诱惑的香味从厨房边上的酿酒坊里飘出来,只是早在琼斯时代那间房子就被弃用了。动物们从来没有闻到过这样的香味,有的动物说只有蒸煮大麦才会发出这样的味道。他们一边贪婪地猛吸这股香气,一边猜测期盼着他们的晚餐就是一锅已经熬好的热乎乎的大麦糊糊。结果不但猜测错误,大麦糊糊没有出现,而且就在那个星期天,新的通告使他们得知,从今往后所有的大麦都得储存起来留给猪们。在此之前,那片苹果园旁边的小牧场早就种上了大麦。不久,就又有消息传出来,现在每头猪每天都能分到一品脱啤酒,拿破仑则会把啤酒装在有英国王冠标志的带盖汤碗里,足有半加仑之多。

动物们现在要忍受的艰难困苦越来越多,但是他们只要一想到现在生活的高贵和自由,他们似乎觉得忍受还是可以继续的,都能说得过去。如今拿破仑颁发又一项新指令,每周举行一次名义上自发的游行,目的单纯就是为了庆祝动物庄园的斗争和胜利,这样一来,歌声多了,演讲多了,游行也多了。一到指定时间,所有的动物都必须放下手头的工作,拿破仑带领着他们,以军队的队形绕着庄园的地界齐步前进,队形是精心排列的,那只小黑公鸡昂首在前面开路,后面理所应当是猪带头,然后是马,其次是牛,再往后是羊,最后是家禽类的。狗永远都在队伍的两边。苜蓿和拳师经常要一边一个用嘴叼着那面标记着蹄子和犄角的绿旗,现在上面又多了一条标语——“拿破仑同志万寿无疆!”等大家围着庄园转完圈后,下一个活动是赞颂拿破仑同志的诗歌朗诵,接着声响器就会把最新的粮食增产数据报告给大家,偶尔还要鸣枪表示庆贺。如果有哪些动物抱怨了,比方说认为这是纯属浪费时间,或者长时间站在寒风中表示出了不满的情绪等等(在没有狗和猪在附近的时候,有个别的动物会这样),绵羊们此刻就会表现出他们是游行活动的忠实拥护者,他们会爆发出气势凶猛的叫声堵住那些动物的嘴,“四条腿好,两条腿坏”被他们重复了一遍又一遍。其实在大部分时候,动物们对这些庆祝活动还是很感兴趣的,毕竟这些活动能让他们真正有了翻身做主人的感觉,他们还是满心欢喜的,他们所付出的,所要忍受的一切是为他们自己争利益,这让他们多少心理平衡些。在闹哄哄的活动中,嘹亮的歌声,浩浩荡荡的队伍,声响器列举的一系列数字,不断的鸣枪庆贺,黑公鸡的啼叫,迎风飘扬的绿旗,参与这一切,看到听到这一切,动物们就暂时忽略了肚子中其实还空空如也,至少部分时间是这样的。

到了四月,动物庄园宣布成立为共和国,那么选举一位总统是当务之急,领袖拿破仑当选是众望所归的,因为候选人只有他一个。就在同一天,一些能够进一步揭露雪球与琼斯相互勾结细节的文件被翻了出来,从这些证据来看,雪球不仅仅是动物们原先所想象的那么单纯了,不单是利用恶毒的诡计妄想输掉牛棚战役及与人类狼狈为奸,公然背离动物主义,与他们为敌。他真正的身份是那支人类队伍的首领,高呼着“人类万岁”可耻地与动物们混战,至于他背上的那道伤痕,有些动物还是记得的,其实是拿破仑同志不顾个人安危咬的。

很快就又到了仲夏时节,消失了数年的乌鸦摩西竟又突然出现在了庄园里。这么多年过去了,他依旧一成不变,不干活,蹲在一根树桩子上,扑棱棱地拍打着他的黑翅膀,要是谁愿意听,就用同以前一样的语气给动物们讲蜜饯山的故事,一旦讲起来,就止不住了。他一脸郑重且煞有介事地说:“看那里,同志们!”还配合动作,他的大嘴朝着天空那里一指,接着继续,“就是那儿,你们看得见的那团乌云的另一边——那座蜜饯山就在那里。那是个幸福的国度,那会是我们的最后归宿,只有在那里,我们这些可怜的动物才能永远地摆脱辛苦的劳作。”他甚至还声称自己去过那里,源于一次机缘巧合的高空飞翔,他说他看见了那里的苜蓿永远那么鲜嫩,还看见了亚麻籽饼和方糖长在树篱上。很多动物对摩西的说法抱着宁愿相信的态度,因为他们现在的生活太苦了,经常饥肠辘辘,疲惫不堪,所以有那么一个比这里要好得多的地方也是合情合理的。只是猪们对待摩西的态度让其他的动物揣测不明,虽然他们对摩西的话报以嘲笑的姿态,认为蜜饯山的说法是一派胡言,但是他们却允许他继续留在庄园,可以不用干活,并且还每天给他发放七分之一升的啤酒作为补贴。

拳师比以前更加卖力地工作,在此期间他已经把蹄掌上的伤口养好了。在这一年中,动物们真实的生活就是像奴隶一样辛苦劳作,除了日常大大小小的事情和风车重建工程以外,在三月份,他们又多了个新的工程,就是给小猪们建造教室。他们很多次饿着肚子还要继续长时间工作,有时真的难以忍受,但是拳师从来没有表现出来,他的步伐从来不会蹒跚踉跄。他的言行举止也没有表现出他的气力有所下降,但是细心观察不难发现:他的皮毛没有以前那样光亮,曾经粗壮的臀部和大腿上部如今也松垮收缩了。动物们也发现了这一点,他们有的说,“拳师会在春草长出来后再强壮起来的”,但是,到了春天,他们说的话却没有成真。他通常会独自一个把一块石头沿着斜坡往矿顶上拉,拼尽全身的力气去承受整个巨石压在他身上的力量,此刻他的嘴唇会一动一动,仿佛在说他的那句座右铭:“我要更加努力工作”,这应该就是支撑着他坚持下去的力量,除此之外,他更多的就是沉默不语。他对待苜蓿和本杰明的“保重身体”的劝告仍然持置之不理的态度。眼看着他的十二岁生日就要来临,这时他更是一门心思地想要在退休之前攒足够的石头,别的事情都不放在心上。

没过多久,拳师就出事了,那是在一个夏天的傍晚,动物得知这个不幸的消息时,天色已经暗下来了,在这之前,有动物看见他又独自一个走出马厩,拉着一车石头去了风车那里。仅仅几分钟,两只鸽子风风火火地飞回来,证实了之前的消息是千真万确的:“拳师倒下了!他侧躺在那里,无法站立起来!”

一大半的动物们不等听完就冲了出去,直奔着建造风车的那个小山包。他们飞跑着来到拳师身边,他就躺在那儿,身子卡在了两根车辕之间,他虚弱得连头都抬不起来,只伸着脖子,低垂的双眼呈现呆滞迷离的目光,汗水浸透了他的整个腹部,一股细流状的鲜血从他的嘴角不断溢出。苜蓿哭泣着跪倒在他身边。

“拳师!你怎么啦?”苜蓿哭喊着。

“没有什么要紧的,应该只是我的肺不行了。”拳师有气无力地回答道,“我明白我最多还有一个月的时间了,我一直盼着能退休,这是我的实话。即使我不在了,我想,我已经攒了不少备用的石头了,你们到时候一定能建成风车。本杰明也老了,也许我们可以做个伴一起退休,如果他们愿意的话。”

苜蓿快速地反应过来:“快,谁去都行,快去告诉声响器,这里出事了,我们必须马上得到救援!”

苜蓿和本杰明留下来照看拳师,其他的动物即刻都跑去院子里找声响器汇报这个紧急的情况。本杰明默默地躺在拳师身边,甩动他的长尾巴来为拳师驱赶苍蝇。过了大概一刻钟的时间,声响器就怀着关心和同情来到了出事现场,他说拿破仑同志也知道了这件事,而且立刻开始着手安排威灵登的医院,把拳师送过去治疗,所以现在他没有过来,就由声响器代表他表达对拳师这样一位最忠诚的朋友遭受如此不幸而感到伤心难过。动物们忐忑不安,除了莫丽和雪球外,他们谁也没有离开过庄园,况且,他们无法相信人类,不愿他们已经受伤的伙伴落入人类的控制中。声响器又开始了他的说服工作,他说庄园里没有兽医,只能在威灵登找到这样的医生来医治拳师,这样才能更快地将他治好,动物们也就这么相信了。半个小时后,拳师稍稍缓过一点劲,吃力地支撑起他虚弱不堪的身体,挣扎着挪回他的马厩,苜蓿和本杰明已经为他铺好了柔软的稻草床。

拳师在以后的两天内一直躺在稻草上,连棚子都没出去过。猪们在浴室的药柜里翻出了一大瓶红色的药片,送来让苜蓿每天两次在饭后喂给拳师吃。到了晚上,苜蓿和本杰明就陪着拳师,苜蓿陪他聊聊天,本杰明则为他驱赶苍蝇。拳师在交谈中提到他并不感到后悔,即使他的身体已经这样。倘若他能完全康复,就希望自己能继续活上三年,在那专为退休员工准备的大牧场的一角里静静地过完余下的日子,那时候,他就第一次真正有空闲时间来学习知识,增强心智。他还说,他一定会穷尽他最后所有的精力学会字母表上除了A、B、C、D以外的二十二个英文字母。

苜蓿和本杰明只有在工作之余才能陪伴拳师。一天中午,动物们正在萝卜地里除草,猪在一边监工,突然,就看见本杰明从棚子那边狂奔过来,嘴里还在大吼大叫,动物们都很吃惊,他们从没有见过本杰明这样激动,事实上这是头一次,就连他奔跑也是动物们第一次见。他扯着嗓子喊叫:“快,快,快来呀!拳师要被他们拉走了!”动物们一听,也顾不得听猪下命令了,大家匆忙扔下手里的活往窝棚那边赶。到了院子里,果然见到一辆大篷货车,车身上写着字,有两匹马拉着,一个头戴低檐圆礼帽的男人坐在驾驶的位置上,他的脸上写满了奸诈狡猾,此时,拳师的窝棚已经空了。

动物们纷纷围到车的四周,异口同声地对着车里喊道:“再见,拳师,再见了!”

本杰明气愤地叫道:“笨蛋!一群笨蛋!”他绕着动物们急得又蹦又跳,又用蹄子不断地跺着地面撒气,“你们没看见车上的字吗?简直蠢透了!”

动物们一下子安静了,场面冷了下来,大家都开始关注车上面的字。穆尔丽开始拼读那些字。但是本杰明着急地一把把他推开,他的声音划开了死寂的气氛:“‘威灵登镇,阿尔弗雷德·西蒙兹,屠马商兼煮胶商,皮革和骨粉的供应商。’你们难道还不明白吗?他们要把拳师拉到屠马场去!”

顿时,动物们都开始号啕大哭,场面一度失控。这时,那个坐在马车上的人突然狠狠地一抽马鞭,那两匹马拉着大篷车轻松地就向外面跑去。全体动物跟在车后面,边掉眼泪边呼喊。苜蓿竭力地挤到最前面。车子开始加速,苜蓿试图跟上马车的车轮,但是她看似粗壮的四肢奋力狂奔也只是达到了慢跑的速度,苜蓿绝望地哭喊着:“拳师!拳师!拳师!拳师!”拳师在这个时候应该是听到了外面的吵闹,他的面孔出现在大篷车后面的一个小窗户上,他的脸上有一道顺着鼻梁生出的白毛。

“拳师!”苜蓿看到他,更加急切地哭喊,声音凄厉,“出来!拳师,你快出来!他们是拉你去送命的!”

所有的动物也都跟着苜蓿哭喊,“出来,拳师,快出来!”大篷车在不断加速,眼见动物们就与大篷车拉开了距离。他们不能判定拳师是否听见了并且懂了苜蓿和他们喊叫的意思。但是不一会儿,拳师的脸从窗户上消失了,紧接着一阵马蹄蹬踹的声音从车内传了出来。没猜错的话,他是想要以此踢出一条生路来。可是天啊,他早已没了往昔的威风,想当初,只需几下,他就可以把这辆车踢成碎片,碎得可以跟火车杆子相比了,现在尽管他用尽了全力,但是大篷车没有一点损坏,马蹄的蹬踢声越来越弱,最终消失了。绝望的动物们转而去哀求那两匹拉车的马,祈求他们能停下来,“兄弟们,同志们!求求你们了,别把你的兄弟拉去送死啊!”但是不管动物们已经多么声嘶力竭,那两匹愚蠢的家伙根本对将要发生的事情一无所知,只是一味地向后抿起耳朵加速奔跑。拳师的面孔再也没有出现在那扇小窗户上。有的动物想起去关五闩大门,但是为时晚矣,瞬间大篷车就冲出了大门,扬起了一路的尘土。就这样,他们把拳师拉走了,从此,拳师就再也没有出现过。

三天后,动物们从通告中得知,拳师在威灵登医院去世了。声响器告诉大家,在拳师弥留之际他一直守在一边,拳师得到的照顾是作为一匹马能够得到的精细入微的最好照料了。

他边说边用一只蹄子擦去挂在眼角的一滴泪,“那是我所见过的最感动的一幕了,我一直在他的床边守着,直到他闭上双眼,在他生命的最后一刻,他虚弱得几乎连说话的力气都没有,但是他还是拼尽全力,在我的耳边说了几句话,对,那应该算是临终的遗言吧,他说他唯一的遗憾就是没能坚持把风车建完。他说:‘前进,同志们!以造反的名义前进,动物庄园万岁!伟大领袖拿破仑万岁!拿破仑同志永远是对的!’”

话说到这里,声响器突然住了口,脸色立刻变得严肃起来,他先是沉默了一小会儿,用他的闪着精光的小眼睛在动物中扫视了一圈,又接着他的发言。

他说,他了解到,关于拳师去镇里治疗有另一个版本的谣传,这个谣传荒唐可笑,用意歹毒。动物们看到了大篷货车上有“屠马商”的标志,就妄加断言拳师被送到了屠马场。他气愤地跳来跳去,还一边甩动着他的尾巴,他说,他难以相信有的动物会这么愚蠢糊涂,根本不用大脑思考。他指责动物们,说仅凭这一点,就说明动物们根本不能真正了解他们敬爱的领袖拿破仑同志。事情的真相是这样的,那辆车以前的确是一个屠马商的,但是现在被兽医院买下来了,他们还没来得及把车身上的字涂改掉,这才造成了大家的误解,是个误会。

听完声响器的这番解释,动物们那颗一直难过的心终于平静了,长长地舒了一口气。接着声响器又讲了很多关于拳师临终前的细节,比如他在医院都受到了怎样无微不至的照顾,以及拿破仑同志是怎样地大方,怎样地不考虑价格给他用贵重的药品。动物们多少心里好受了一点,他们之前一直对拳师的死耿耿于怀,现在听到他是这么安详地走,就感到心宽了许多,所以对于声响器的话没有丝毫的怀疑了。

那个星期天早上的会议,拿破仑竟然亲自出席了,他宣读了一篇哀悼拳师的简短的悼词。之后他对动物们解释说,因为某些原因,拳师同志的遗体已经不能拉回来安葬在庄园里了。但是根据他的指示,已经用庄园主院子里的月桂花编成了一个大花圈,放在拳师的墓前。猪们还准备就在这几天专门给拳师开一个追悼宴会。最后,拿破仑引用了拳师生前的座右铭结束了他的讲话,“我要更加努力工作”和“拿破仑同志永远是对的”,他希望所有的动物都把这两句话当成是自己的座右铭,激励自己,并认真贯彻落实。

到了已经确定好的追悼宴会的那天,庄主的院子里又来了一辆从威灵登驶来的马车,这次是杂货商送来了一个大木箱。到了晚上,院子里又传出了闹哄哄的歌声,之后,还伴随着激烈的吵架声音,一直持续到夜里十一点多,在一阵玻璃破碎的可怕声音后,归于平静。第二天到了中午,庄主院子里还是一片沉寂。与此同时,一个小道消息快速地传出来,猪们不知道从哪又搞到一笔钱,买了一箱威士忌给他们自己享用。
