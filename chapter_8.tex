\chapter{第八章}

直到几天以后,动物们对那种成批杀害同伴引起的恐慌差不多快要完全退去时才想起,或者应该说自己认为记得,戒律中第六条有规定:“凡动物一律不准杀害其他动物”。他们觉得之前发生的杀戮事件违背了戒律所传达的精神,但是再提及这个话题时,他们却不希望被猪和狗听到。苜蓿请求本杰明念第六条戒律给她听,本杰明像往常一样,对待此类问题的态度依然是拒绝掺和。没办法,苜蓿只能又找到穆尔丽。穆尔丽是这样给她念的:“凡动物一律不准杀害其他动物毫无缘由。”动物们对后面的几个字,似乎不知道为什么没有一点印象。如果依照戒律上所写的,那么处死那些与雪球沆瀣一气的叛徒是有理由的,完全没有违反戒律。

与前些年相比,动物们在这一年干得更为辛苦。又要重建风车,又不能丢下庄园里的那些常规工作,更艰巨的是,重建风车不但要加厚墙体,并且是原来的两倍,还要在原先规定的日期内,这样动物们背负的任务可想而知是多么艰巨。动物们已经多次感觉到,与琼斯时代相比,现在的劳动时间长,而且吃也不比那时好。声响器在每个星期天早上出现,他用蹄子夹着一张长长的纸条,纸条上写着各类食物产量增加的数据,根据种类的不同,有的增加了百分之二百,有的增加了百分之三百,他把这些信息传达给动物们,动物们找不到怀疑他的理由,也就相信了,反正他们现在也已经记不清楚造反前是个什么样的状况了。即便不否认这些数据,但是对于动物们来说,如果能多给他们分点吃的,他们宁愿那些数字能减少些。

拿破仑现在露面的次数越来越少,动物们两个星期能见他一面就算不错了,他让声响器和另外一头猪传达他所有的命令。他现在每次出现,都会有一只小黑公鸡排在队伍的最前面来开道,主要充当号手的角色,他先要响亮地啼叫一遍“喔—喔—喔”,用来提醒大家拿破仑要发言了。据说拿破仑所住的地方,就是在庄主院子里,也是单独住在套房里,跟别的猪是分开的。他单独用餐,通常他使用的餐具是那些陈列在客厅玻璃橱柜里的带有英国王冠标志的德比瓷器,并且在他用餐的时候,有两条狗会在旁边侍奉着。他还宣布了一个命令,就是每年在他生日那天,也要像另外两个纪念日一样,鸣枪以示庆祝。

如今的拿破仑,就连称谓也有所改变,在任何时候谈到他时不能再随随便便简单直呼为“拿破仑”了,而是要在他的名字前面加上“我们的领袖”,以示对他的尊敬,其他的猪给他赋予了很多光鲜的头衔,例如“动物之父”“人类克星”“羊的保护神”“鸭子的亲朋”等。声响器在每次的演讲里都满怀深情地向动物们传达着拿破仑的种种:他的智慧是多么卓越超群,他的心肠是如何仁慈善良,他是在多么用心地爱着每个动物,尤其是对那些仍在别的庄园里忍受人类奴役和歧视的不幸动物,他倾注了更多的爱。在庄园里有了一项新的惯例,不管是新得了一个什么成就,或者是发生新的什么幸事,大家都会不约而同地把荣誉归功于拿破仑。庄园里随处都可以听见这样一些谈话,一只母鸡对另外一只母鸡说:“在我们的领袖拿破仑同志的领导下,我六天产了五只蛋”,或者在池边饮水的两只牛突然感慨道:“这水真甜,这全是拿破仑同志的英明领导所导致的。”综合了庄园里现在动物们的普遍表现和充分感受,小不点创作出了一首名为《拿破仑同志》的诗,全诗如下:

孤儿的亲朋!

幸福的源泉!

恩赐食物的上帝!

您那镇定威仪的双眼

犹如天上的太阳,

每当我仰望着您

啊!我心中满怀激情

拿破仑同志!

是您赐予

您那一切生灵的挚爱,

每日饱餐两顿,

还有那洁净的干草可以打滚,

所有的生灵不论大小,

都能在窝棚中安然入睡,

仰仗您的守护,

拿破仑同志!

我若有一头襁褓中的小猪,

不等他长大,

哪怕他才奶瓶般大,擀面杖般长,

他就应该懂得

永远对您忠心耿耿,

对,

他发出的第一声尖叫一定是

——“拿破仑同志!”

这首诗让拿破仑感到前所未有的满意,他命令手下把它同样刻在大谷仓墙上,“七诫”的对面。在诗的正上方,声响器用白漆描绘了一幅拿破仑的侧身肖像画。

这个时期,那堆木材仍旧堆在院子里,虽然温普尔极力从中牵线,拿破仑还是迟迟不作决定,继续和弗雷德里克及皮尔金顿周旋谈判。相较于皮尔金顿,弗雷德里克更急于得到这批木材,但是他又想占便宜,不愿意出合理的价钱。一个谣言又在这个时期出现:雪球就藏身在平彻菲尔德庄园,弗雷德里克先生伙同他的伙计正在计划一个偷袭动物庄园并且再次摧毁风车的阴谋,因为风车的建造让他们心生妒火。在仲夏的一天,三只鸡主动交代了一个让所有动物再次震惊的罪行,他们曾在雪球的教唆下,密谋过要杀害拿破仑同志。毫无疑问,这三只鸡被立刻处决了。在这之后,一系列的为了拿破仑同志的安全的新防范措施被执行,首先在夜里,他的每个床脚边都有一只狗守卫着,其次为了防止食物被下毒,一只名为红眼睛的小猪负责在拿破仑用餐前,品尝他要享用的每一道食物。

差不多也是在这个时候,动物们得到消息,拿破仑终于决定把那堆木材销售给皮尔金顿先生,同时他还打算同福科斯伍德庄园就某些产品签订一项长期的交换协议。温普尔先生在这次的双方交易中出了很大的力,拿破仑先生和皮尔金顿先生的关系也差不多变得相当友好了。但是动物们因为皮尔金顿是人类,怎么也不能对他产生信任,但是弗雷德里克更让他们不信任,对于这个人,动物们既怕又恨,所以,相比之下,与皮尔金顿打交道是他们更好的选择。风车工程一直都在紧张有序地进行着,随着夏末的来临,风车的建造也接近了尾声,那个关于要发生突袭的传言也愈演愈烈。传言说弗雷德里克买通了地方的官员和警察,要率领二十个完全武装的人类来袭击动物庄园,这样一旦袭击成功,他就能拿到动物庄园的地契,并且那些官员和警察也不会管的。除此以外,还有更为恐怖的信息不时地从平彻菲尔德庄园传出。费雷德里克那个冷酷无情的人类用他的动物们来进行各种各样残忍至极的实验:他活活地鞭笞死了一匹衰老的马;他故意不喂他的牛,让他们忍受饥饿;还把一条狗扔进火炉里,致使他活活烧死了;他把剃刀刀片绑在公鸡的爪子上,让他们互相打斗以供他取乐。动物们在听到这些让他们震惊的消息后,怒火中烧,根本无法忍受他们的同伴、同类竟然遭受如此的迫害,愤怒的情绪使他们斗志昂扬,他们不停地叫嚷着,想要全体出动去进攻平彻菲尔德庄园,拯救他们的伙伴,惩罚那里那些可恶的人类,把他们统统撵走。但是声响器此刻冷静地告诫大家,要充分地相信拿破仑同志的智谋策略,万不可草率行动。

但是动物们高涨的战斗热情并没有因此而消退下去,反对弗雷德里克的热潮一浪高过一浪。拿破仑终于在一个星期天早上出现在了大谷仓,他向动物们解释了他曾经的想法,他保证说从来都没有过把木材卖给弗雷德里克的想法,因为他早知道那个人是无赖,和无赖交易会有损他的尊严。他还说为了鸽子们的安全,以后出外散播造反消息时,禁止在福科斯伍德庄园的任何地方落脚。他们以前的“打倒人类”的口号现在正式更改为“干掉弗雷德里克”。到了夏季的尾声,收获小麦的时候,动物们发现了雪球干的又一件坏事,他曾在某个夜晚偷偷潜回庄园,在粮种里掺进了杂草的种子,致使杂草混迹于收获的小麦中。一只公鹅向声响器坦白了他曾是这个阴谋的参与者,之后他就立刻吞噬剧毒浆果自杀身亡了。关于雪球的那个至今他们还相信的“一级动物英雄”的荣誉,现在动物们更进一步地了解到,那就是一个神话故事,那是雪球自己在牛棚战役后散布的假消息。真实的事情其实是雪球在战争中表现得很懦弱,遭到了大伙的谴责,这个家伙压根就没有被授予过任何勋章。一些动物们再次表现出疑惑,但是声响器又再次使他们打消了这些疑虑,使他们相信是他们自己的记忆出了问题。

随着秋天的到来,庄稼开始了收割,动物们只有更加努力工作,因为风车的建造不能中断,风车在动物们竭尽全力的拼搏下终于完工了。但是建成的只是风车的主体结构部分,接下来还需要有机器设备,温普尔现在正在为设备的购买与人四处谈判。风车这项伟大的工程历经了多么坎坷的过程,所走过来的每一步都是那么艰辛,他们的经验是如此匮乏,设备是如此简陋,运气又是那么不好,还有雪球在其中捣乱,但是最后,它终于还是在动物们的集体努力下竣工了,并且是如期的,没有晚一天一刻。动物们殚精竭虑地完成风车后,满怀着无比喜悦和骄傲自豪的心情一圈一圈围着风车转,这是他们倾注了满腔的热情,挥洒了无尽的汗水取得的劳动成果,在建造的过程中数不清战胜了多少困难和挫折,此时的风车在他们的眼里比第一次建成后更加好看了,也更加牢固了,因为墙体加固了一倍以上的厚度,除非使用炸药,否则谁也不能摧毁它了。此刻他们幻想着在装完所有的设备后,他们的生活就会伴随着风车翼板带动发电机运转后而发生翻天覆地的变化。他们越想越兴奋,疲劳感顿时烟消云散,抑制不住的激动使他们连连发出胜利的呐喊,他们止不住地围着风车转圈圈。拿破仑又以他的标准队形亲临了风车现场,他以他个人的名义对为风车贡献力量的动物们表示了祝贺,并且宣布,风车命名为“拿破仑风车”。

过了两天,拿破仑命令全体动物在大谷仓集合,他给大家开了一个特别的会议。会上,他正式宣布把木材卖给了弗雷德里克,并且,明天弗雷德里克就会带领车队来这把木材拉走。动物们被这个决定震惊了,大家惊讶得说不出一句话。他们被告知,其实在这段时间,拿破仑只是在表面上同皮尔金顿非常友好,其实在私底下一直同弗雷德里克来往密切,并且早就跟他达成了秘密的协议。

这样一来,就与福科斯伍德庄园断绝了一切来往,随后,他们就发出很多侮辱性的信件给皮尔金顿。那些散播信息的鸽子再次被提醒这次要避开的是平彻菲尔德庄园,还有就是把“干掉弗雷德里克”的口号换成了“干掉皮尔金顿”。拿破仑又向动物们作了保证,那个据说动物庄园会遭受到袭击的消息完全是子虚乌有的,此外,谣传弗雷德里克虐待动物是完全无限夸张了的说法,捏造这些传言的很有可能就是雪球与他的同伙们。总之,雪球一直躲在福科斯伍德庄园里,他压根就没有去过平彻菲尔德庄园,更别说躲在那里了。他早就是皮尔金顿门下的一个名副其实的食客,在那里过着骄奢淫逸的生活。

拿破仑表面上看起来对皮尔金顿非常友好,这样迫使急切想买木材的弗雷德里克提高了十二英镑的报价,猪们对拿破仑的这个精明的计谋感到无比崇拜。声响器还说,拿破仑真正精明的是他对每个人类都不信任,即便是对弗雷德里克也是如此,原本弗雷德里克准备要用一个叫作支票的纸质东西来支付木材的钱,其实就是一张纸上写着类似保证会付款之类的承诺,但是英明的拿破仑同志岂能被这样的把戏骗了,他要求弗雷德里克用真正的面额是五英镑的钞票来支付货款,而且必须是先交钱,然后才能把货拉走。现在弗雷德里克先生已经交清了所有的货款,这些货款正好够买风车所需要的所有设备。

很快木材就被拉走了。大家在木材被全部拉走后又聚集在大谷仓开了一场会议,这次的会议很特别,只是单纯地让动物们检阅那些弗雷德里克支付的钞票。会议期间,拿破仑面带笑容,满心欢喜地端坐在那个平台上的一张草垫子上,他的胸前佩戴着他的两枚勋章,钱就整齐地码放在他身边的一个从庄主院子里的厨房拿来的瓷盘子上。动物们列队依次缓缓地走过,目不转睛地盯着,仿佛要一次看个够。拳师还伸过脑袋,用鼻子使劲嗅了嗅那堆钞票,那些轻薄的白色花纸片在他呼出的气体的搅动下,发出轻微的哗哗声。

三天后,这次温普尔的到来引发了一场可怕的骚乱。只见他脸色苍白,如死人般,老远就看见他骑着自行车沿着小路一路飞奔,到了庄园把自行车一扔,就急忙跑进拿破仑的屋子。不大一会儿,一阵仿佛要窒息了的狂怒的吼叫声从拿破仑的房间里传出来,响彻了整个庄园。关于出事的消息就如无法控制的野火迅速蔓延了整个庄园。钞票是假的!就是说弗雷德里克没有花费一分一毫就拉走了所有的木材。

拿破仑立刻把所有的动物召集来,用极度可怕的声音给弗雷德里克判处了死刑。他命令动物们,只要抓住了弗雷德里克,就把他活活地煮死。同时,他还发出了警告,在这次背信弃义的欺骗事件之后,那个奸诈的弗雷德里克很可能会随时伙同其他无耻同伴发动对动物庄园的蓄谋已久的袭击。因此,在他的部署下,每个通向庄园的路口都已经布下了岗哨。此外,一封和好信由四只鸽子送去了福科斯伍德庄园,希望能与皮尔金顿先生和好如初。

敌人的进攻就在第二天清晨开始了。那时正是动物们吃早餐的时间,一个哨兵飞奔着来报告,说弗雷德里克带领着他的伙计们已经走进了五闩大门。动物们听到汇报后立刻起身去应对战斗,毫不畏惧。这次这些人一共来了十五个,带着六支枪,动物们在这次战斗中可没有像在牛棚战役中那么轻易取胜,在他们相距还有五十码的时候就遭到了射击,可怕的枪声还有密如雨点的子弹让他们无力抵挡,拿破仑和拳师不得不努力让他们重新聚集在一起,可是很快随着更多的动物负伤,他们又被击退了。没办法他们只能退回到庄园的窝棚里躲避起来,从墙缝和木板的结疤孔里向外小心翼翼地张望着。此刻整个大牧场包括风车在内都落入了那伙人手里。拿破仑这时也有点慌神了,他默不作声,来回地踱着步子,尾巴僵直着还不停地抽搐。大家都用期盼的眼神遥望着福科斯伍德庄园,只要皮尔金顿能率人前来援助,今天就很可能转败为胜。那四只在前一天被派去送信的鸽子在这时回来了,却只带回了一张皮尔金顿亲自用铅笔写的小纸条,上面只有四个字:“你们活该。”

与此同时,再看向大牧场那里,弗雷德里克那一群人已经停在了风车周围。动物们一边窥视着那边的动静,一边惊慌绝望地小声议论着。只见那群人中有两个拿出了一根钢钎和一把大铁锤。看那架势,他们是要准备拆除风车。

拿破仑大声地安慰大家说:“他们那么做是徒劳的,我们把风车的墙体建得足够厚了,一个星期内也休想拆了它。同志们,都勇敢点。”

那两个人拿着大锤和铁钎在风车靠近底部的墙面上凿着孔。本杰明一直全神贯注地盯着那些人的一举一动。看到最后,他的神态已经开始带有戏弄的表情,且还慢慢悠悠地上下摆动着他的长长的嘴巴。

然后他开口了:“我早就猜到他们会这么干,你们知道他们正在做什么吗?不一会儿,他们就会把炸药塞进那些他们凿好的小孔里。”

动物们吓坏了,但是他们没有好的办法,现在冒着生命危险冲出窝棚不太可能,只能静观其变。很快就见那些人四下跑开了,随之而来的就是一声几乎击穿耳膜的爆炸声。震得鸽子们立刻飞上了天,除了拿破仑,其他的动物都立刻卧倒了,把脸藏了起来。过了一会儿,他们爬起来,再次看向大牧场时,只见原来风车的位置上笼罩着一团巨大的黑色烟雾。微风慢慢吹淡了烟雾后,露出了空旷,风车已经消失了!

这一幕深深地刺激了动物们,这种可耻的行为使他们狂怒不已,立刻热血沸腾,之前感到的所有胆怯和绝望,此刻完全被勇气淹没了。突然,一阵复仇的呐喊声响起,在没有命令的情况下,动物们已经等不及了,不约而同地朝敌人们径直扑过去。此次,枪林弹雨挡在他们前面,他们也不再理会,不再退缩,任由那些无情的子弹擦着他们的脑袋呼啸而去。这场战斗残酷,惨烈。动物们冒着冰雹般密集的子弹冲到了他们眼前,他们就开始用棍棒一通乱舞,同时还用沉重的靴子猛踢。动物们几乎都挂了彩,拿破仑也不例外,即使他一直在后方指挥作战,他的尾巴尖也被子弹削去一小截,还有一头牛、三只羊和两只鹅壮烈牺牲了。人也不是毫发无损,他们中有三个人被拳师的蹄铁踢破了脑袋;还有一个人被一头牛用犄角顶破了肚皮;另外还有一个人差点被杰西和蓝铃铛扯掉了裤子。拿破仑命令他那九只贴身狗护卫,利用树篱的掩护迂回包抄,敌人被突然出现在他们右翼的这群狗吓了一跳,他们气势汹汹的狂吠更是把人们吓得魂都飞了。人类开始发现快要被包围了,有危险,弗雷德里克下令同伙们趁现在还有退路,就开始撤退了。贪生怕死的敌人很快溃不成军,他们纷纷逃命。动物们一路穷追不舍,一直追到地界边上,最后还在人类慌不择路的情况下越过那片荆棘树篱时,狠狠地踹了他们几脚。

即使这场战争最后以动物们取胜为结局,但是他们个个鲜血淋漓,精疲力竭。他们一瘸一拐地往回挪着步子,路过草地,看到横躺在地上气绝了的同伴,有的动物悲伤得泪流满面。他们在风车的位置不由自主地停住了脚步,大家表情肃穆。原本矗立的风车不见了,是的,一点痕迹都没有了,那是他们艰辛拼搏才得来的成果,现在就连地基也被破坏了一部分。现场都没有落石,炸药的威力把它们不知送到了几百码外,这一次要想重建,可不像上次那么简单了,没有坍塌的石头利用了。现在风车的位置就像从未曾有过风车一般。

动物们哀叹完之后继续往庄园走,快要进庄园时,莫名其妙在战争中消失的声响器此时兴高采烈地一蹦一蹦地前来迎接他们。此时此刻,他的脸上挂着满意的微笑,他不停地晃着他的尾巴,与此同时,从窝棚那边传来一阵庄重的鸣枪声。

“开枪做什么?”拳师不解地问。

“当然是庆祝我们的胜利啊!”声响器欢快地叫道。

“胜利?什么胜利?”拳师继续问。他的膝盖还在流血,一个马蹄铁也在战斗中丢失了,他的蹄子裂开了一道大口子,他的后腿挨了足足十二颗子弹。

“什么胜利?难道我们不是刚刚把侵略者赶出了我们的地盘——神圣的动物庄园吗?同志们!”

“可是我们辛辛苦苦建造了两年的风车被他们毁了,整整两年的劳动成果就这么没了!”

“这有关系吗?只要我们愿意,我们就可以建造一座,六座都没有问题。你们还没有意识到我们干的这件事是多么伟大啊!得益于拿破仑同志的英明领导,我们脚下的这些曾经被侵略的土地再次被我们一寸一寸夺回来了!”

“可是我们夺回来的只不过原先就是我们的。”拳师说。

“这就是胜利,毫无疑问!”声响器说。

他们一瘸一拐地进到了院子里。拳师的那条挨了枪子的腿剧烈地疼痛着。但是他还在想着重建风车,他明确地知道摆在他面前的工作,这次要从打地基开始,又将是一场耗心耗力的艰苦劳动,他想象着为这项任务迅速作好了准备。但是他第一次考虑到他已经十一岁了,身体虽然还强壮,但是依旧不能与当年相提并论了。

即便刚才与声响器争论了这次战争的意义,但是在动物们看到迎风飘扬的绿旗,听到了再度响起的猎枪声,这次足足响了七下,又得到了拿破仑领袖表彰他们英勇行为的贺词后,他们似乎接受了,这场战争他们毕竟还是取得了重大的胜利。大伙为在战争中壮烈牺牲的同伴们举行了一个隆重的葬礼。拿破仑亲自走在送葬队伍的最前面,拉着灵车的是苜蓿和拳师,灵车是临时改装的四轮运货车。对胜利的庆祝活动整整持续了两天,其中包括歌咏,演讲,以及必不可少的鸣枪等。每个动物都给予了食物奖励,也算是特别的礼物,家禽类的是两盎司谷子,每只狗发的是三块饼干,剩下的每个动物被赐予的是美味的苹果。这场战争,拿破仑把它命名为“风车战役”,同时创立了一个新勋章——“绿旗勋章”,并把它颁发给了自己。就这样,钞票事件带来的不幸早就被战争胜利带来的兴奋给掩埋了。

猪们在庆祝完胜利的几天后,无意间在庄园主的地下室里发现了一箱威士忌,在他们刚入住这里的时候可是没有搜索到。当天晚上,令每一个动物都感到惊奇的是,一阵嘹亮的歌声从庄园主的院子里传了出来,并且还有《英格兰兽》的调子。大概是九点半的时候,拿破仑被有的动物发现戴着一顶旧的圆顶礼帽从后门走出来,然后飞快地围着院子跑了一圈后又闪进了门内消失了。第二天的庄主院子里静悄悄的,没有一头猪走动,直到九点多的时候,只有声响器出现了,他看起来状态不怎么好,步履缓慢,神情呆滞,垂头丧气,浑身上下就连尾巴也有气无力地耷拉着,整个一个病入膏肓的样子。他召集了所有的动物,悲哀且沉痛地告诉了他们一个不幸的消息——拿破仑同志病危了!

哀悼的哭声立即响起来。动物们无比关心拿破仑同志的身体状况,庄园主院子的门外铺着干草,他们经过时都踮起了蹄尖。他们的双眼饱含着泪花,大家互相询问,万一伟大的拿破仑领袖不在了,他们可怎么办?之前雪球要谋害拿破仑的消息此刻又传开了,这次他费尽心机却是成功地把毒药掺在了拿破仑的食物里。十一点,声响器作为拿破仑的发言人,代他宣布了他在弥留之际的最后一项措施,也是一道神圣的法令:饮酒者必须处以死刑。

好在到了晚上,拿破仑的病情就有了好转,到了第二天早上,声响器就告慰大家,拿破仑正在康复的过程中,很顺利的是,到了晚上,拿破仑的身体就正常了,并且能开始工作了。又过了一天,有动物透露,拿破仑曾指示温普尔到威灵登购买一些关于蒸馏及酿造酒类方面的小册子。过了一个星期,拿破仑下令,把苹果园那边的小牧场加以翻耕,但是之前是把这个小牧场定为退休动物的放牧场的。现在拿破仑说那个牧场的地力已经耗尽了,需要重新翻种,没过多久,拿破仑的真实目的被揭晓了,他其实想在那里种大麦。

在这期间,在一个月明星稀的夜晚,发生了一件怪事:大概是在夜里的十二点多了,一声巨响,听着应该是什么物体跌落的声音,动物们纷纷冲出自己的窝棚来寻找声音的发源地,他们不可思议地看到声响器昏迷不醒地倒在地上,他的旁边横着一张断成了两半的梯子,离他很近的大谷仓墙上,写着“七诫”,他手边上放着一盏马灯,一把油漆刷子,还有一罐泼了一地的白漆。几条狗看到此景后立刻把声响器包围了起来,等到他能够自己走路了,就护送他回到了庄主的院子里。动物们对这是什么情况迷糊不解,只有老本杰明上下摆动着他的长嘴,一副心知肚明的样子,似乎那其中的秘密他早就领会了,但是他就是这样,什么都不会说出来。

穆尔丽在几天后自己给自己读七诫时发现,原来还有一条戒律她之前记错了,其他的动物也都记错了,他们原本记得第五条中写的是“凡动物一律不准饮酒”,其实真的戒律要多出两个字,应该是“凡动物一律不准饮酒过量”。
