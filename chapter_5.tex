\chapter{第五章}

随着冬天不知不觉地到来,大家越来越不喜欢莫丽。不只是讨厌她总是在干活的时候来得最晚,还总是以睡过点作为借口,她还常常嘟囔一些奇怪的病痛。不过,这些并不影响她吃饭,一口都不会少吃。她经常以种种借口不干活,自己躲到饮水池边,呆呆地凝视着自己在水中的倒影。甚至还有一些传言比这更为严重的。有一天,莫丽一边晃着她的长尾巴,一边嚼着草根,慢悠悠地遛到院子里,苜蓿把她拉到一边。

“莫丽,”她说,“我要跟你说一件很严肃的事情。今天早上,在你查看我们庄园和福科斯伍德庄园隔着的那段树篱时,有一个皮尔金顿先生的手下就在树篱的那一边。尽管我离得远,但我还是看见他跟你说话了,他还抚摸了你的鼻子。出了什么事,莫丽?”

“不可能!他没有摸我!我也没让他摸!这是不可能发生的!”莫丽大声地反驳着,同时她的蹄子不安地扒拉着地上的土。

“莫丽!直视我,你敢用名誉保证,那个人没有摸过你的鼻子吗?”

“这不是真的!”莫丽反复说道,但是却不敢抬头看苜蓿。然后,她就朝着田野跑去了,显然是想逃避。

突然一个念头在苜蓿心中闪过。她没有和其他动物商量,立刻径直跑到莫丽的厩棚,在干草堆里翻了翻,很快,露出了一小堆方糖和几束各种颜色的饰带。

三天后,莫丽不见了,接连几个星期都下落不明。直到后来从几个鸽子口中才得到一点关于她的消息。鸽子说见到她的时候是在威灵登的另外一边酒吧外,当时她的身上正套着一辆红黑色的双轮马车。嘴里还吃着酒吧老板喂给她的方糖,鼻子还享受着酒吧老板的抚摸,酒吧老板是一个穿着格子马裤和高筒橡胶靴的红脸胖子。貌似她的毛发被修剪一新,一条鲜红的饰带作为装饰绑在她的额头上。鸽子们说莫丽对现状很是满意,脸上洋溢着骄傲。从那以后,动物们就再也没有提过莫丽。

很快到了一月份,天气很是恶劣。泥土冻得邦邦硬,地里的活都停了。但是大谷仓里的会议却比以往多了起来,猪们忙于筹划下一个季度的工作。大伙都已认同由动物们中智商最高的猪来决定庄园里的大政方针,即便他们的决策最后还是要在大多数同意的情况下才能执行。雪球和拿破仑还是一如既往地为所有的安排争吵,致使一切都不是很顺利。每一个能发生争执的地方,他俩都没错过。假如一个提出应该播种更大面积的麦子,另一个就一定要求用更大的面积播种燕麦;如果一个说某某地方非常适合种植卷心菜,那另一个肯定就说那里只是一块只能长活根茎作物的烂地。他俩的追随者旗鼓相当,一些激烈的辩论不可避免。在每周的例会上,雪球侃侃而谈,绝大多数动物都很赞同他的观点。而拿破仑擅长于利用会议的中间休息时间用游说的方式为自己拉票。绵羊对于这种方式尤为支持。在以后的不论何时何地,绵羊们都会经常高呼“四条腿好,两条腿坏”,故意在大会上捣乱。并且,大家渐渐发现,尤其在雪球发言的高潮阶段,更容易听见他们咩咩地叫喊“四条腿好,两条腿坏”。雪球曾经看过一些过了期的《农场主与畜牧业》杂志,当然是从农场主的屋里找到的,他对此进行了深入的钻研,学习使他的脑袋里有了很多的新点子和改革的设计。他一谈起农田排水,饲料保鲜还有碱性煤渣这些术语什么的,就滔滔不绝啊。此外,他还设计出了方便施肥的系统,这个系统很复杂,但是很方便,可以把动物们随时随地排放的粪便直接通到农田,省了再用马运输的劳力。拿破仑虽然没有实质性的贡献,但并不妨碍他继续跟雪球作对,他说雪球的这套方案听起来天花乱坠,但其实根本达不到最后的效果。看来,他是等着看雪球的好戏了。在他们俩所有的争吵中,最激烈的还数关于风车一事的争辩。

庄园里的窝棚距离那狭长的大牧场不远,大牧场那儿有一个能俯瞰整个庄园的小山包。雪球经过实地考察后得出结论,那是建造风车的绝佳之地。风车可以带动发电机,发电机发了电可以为整个农场供电。电可以为窝棚照明,也可以在天冷的时候用来取暖,还可以用在一切先进的电器上为生产带来方便,比如电锯、铡草机、切片机,还有电动挤奶器等等。动物们之前从没有听说过这些名词(因为这是一个有年头的庄园了,机械都是最原始的)。雪球绘声绘色地给动物描绘有了这些先进机器之后的生活场景,动物们很快就把自己想象进去了:他们只需要悠闲地在地里吃草,或者阅读谈天,做自己想做的事情,一边的机器可以帮他们干所有的活。

雪球在短短的几个星期中就拿出了风车的设计方案。原本属于琼斯先生的三本书:《实用家居一千条》《自己动手建房舍》和《电工入门》给了雪球很多机械设计方面的参考。曾经存放孵卵器的那间屋子现在是雪球的工作室,因为里面铺了木质地板,很光滑,很适合画图。他常常把自己关在那里,一干就是好几个小时。他把书本打开用石块压好,用蹄子上的两趾夹住粉笔,绘几下图,看一眼书,还会来回地走上两圈用于思考,偶尔发出些兴奋的哼唧声。整个设计图已经进入到最复杂的阶段,大量的曲柄和齿轮要错综复杂地连接到一起,大半个地板被图占据了,其他动物虽然看不明白,但是留给他们的印象还是很深刻的。动物们每天都来看雪球画的图,一天最少来一次。鸡和鸭也不例外,他们还很自觉地不去踩踏那些粉笔画的线。只有拿破仑从不关心雪球这边的进展,他甚至从来都没有来看过图。他最初就反对制造风车。出乎意料的是,一天拿破仑竟然来到雪球的办公室查看设计图。他肥胖笨重的身子在屋子里转过来绕过去,设计图上的每一处都被他看了个仔细,还用鼻子凑上去使劲地嗅了嗅,完了他眯着眼站在图边沉思了一下。突然就抬起后腿,在图上撒了泡尿,但是并没说什么就走了。

在风车这件事上庄园内部产生了严重的分歧。雪球从一开始就不否认风车是个大项目,制造起来会很麻烦的,要筑墙就得采石头,还需要制造风车的那个翼板,还得准备发电机和电缆(但是如何得到这些东西,雪球只字没提)。虽然面临诸多困难,雪球依然信心满满,他宣称不用一年就能建好风车,工程完成后就能给大家带来很多福利,肯定会节省大量的劳动力,这样,动物们只需每周工作三天就行了。对立面的拿破仑很快就反驳了,他认为现在最应该做的是多生产食物,如果把精力都放在制造风车上了,生产怎么办?食物不够,大家怎么活命?于是两个派别应景而生,一派打着的口号是“拥护雪球和每周三天工作日”,另一派的口号是“拥护拿破仑和食物满槽制”。唯一的中立者是本杰明,因为他谁的话也不信,他认为食料不会有量的增加,风车也不会像雪球说的那么神通广大,所以有还是没有,于他来说没有关系,日子继续像以往一样过,有不如意在所难免。

风车的事情大家依旧争执不下,但是庄园的防御也是问题。虽说牛棚战役动物们大胜,但是琼斯先生他们逃走了,他们打了败仗的消息传遍了整个英格兰,使他们颜面尽失,其他庄园的动物得到这个消息后更加难以驾驭了,种种状况表明,人类有理由,有充足的理由,有可能,完全有可能发动一场更为惨烈的反攻来帮助琼斯复辟庄园。就连防御的问题解决方式,拿破仑和雪球也不出意外地意见不合。拿破仑认为应该武装保卫,多弄一些武器,进行训练学习,自己保护自己,否则就等于坐以待毙。而雪球则觉得应该放出更多的鸽子,散播更多关于动物庄园的消息,让更多的动物反抗造反,使当前的形势更加风起云涌,自卫就不需要了。动物们在这个问题上摇摆不定,分不清哪个对哪个错,他们觉得好像谁的话都有道理。后来他们发现,只要谁在讲话,他们就相信谁。

雪球的设计图终于完成了。于是关于是否建造风车的议案顺理成章地又被正式搬上了星期天的碰头会上。动物们集合完毕,会议正式开始。首先雪球开始作陈述,他把建造风车的好处一条一条摆出来,尽管中间被羊发出的咩咩声打断了n多次,但是他对要建造风车的态度还是很坚定的。接着,拿破仑提出了反对意见,他旁敲侧击地说风车只可能是幻想,大家不能当真,不要拥护这个华而不实的议案。说完他就坐下了,丝毫不关心群众的反应,前后不过半分钟。雪球一见拿破仑结束了发言,就迫不及待地站起来,制止住还要咩咩叫的羊,开始再次慷慨激昂地陈述自己的态度,呼吁大家要支持建造风车。本来之前大家分成了两派,分别支持两头领袖,但现在被雪球的一番话语弄得纷纷倒戈。雪球用他带有魔力的话语带领着动物们进入到了美好的憧憬里——庄园里,动物们会因为有了风车摆脱了又脏又累的生活,此刻不仅仅是有铡草机、萝卜切片机了,还会因有了电而使脱粒机、犁、耙、碾子、收割机还有捆扎机依次运作起来。电还能使生活更为舒适,会为窝棚照明,提供冷热水,天冷还能用电暖气取暖。等他带着大伙畅想完了,表决会何去何从已经很明显了。就在这个当口,拿破仑站起身来,丢给雪球一个奇怪的眼神后,发出了一声前所未闻的尖叫。

突然,会场外响起一阵令人毛骨悚然的咆哮声。随即,蹿进来九只强壮的狗,个个都戴着铜式项圈,瞅着雪球就扑了过去。雪球条件反射地跳开了,勉强躲过了被撕咬的厄运,但是不容他反应,九只狗又扑过来,雪球只能奔出门外,九只狗依旧穷追不舍。动物们被这突如其来的状况搞傻了,除了惊恐,就只会站在门口观望这场追逐。雪球一路飞奔过狭长的牧场,朝着大道跑去,他拼了命地奔跑,狗们紧追不舍,突然他很不幸地滑倒了,就在这千钧一发的时刻,求生的欲望爆发了,雪球重新爬了起来,继续更加拼命地跑,狗们又一次逼近了,其中一只狗差点咬到了他的尾巴,幸而他甩开了。接着他一个奋力冲刺,赶在狗嘴碰到他之前,从树篱的一个豁口蹿了出去,转眼就消失了。好险,就差几英寸,最后雪球算是犬口余生。

动物们心惊胆战地回到大谷仓,没有人敢出声。不一会儿,那些狗叫嚷着跑了回来。起初,大家谁也想不明白这些狗是从哪冒出来的,但很快大家就想起来了,那就是杰西和蓝铃铛产的那九只狗崽子,一直被拿破仑偷偷养着。这九只狗虽然还未完全长大,但个头和成年的相差无几,个个面相凶狠,犹如饿狼。大家还发现,他们始终紧挨着拿破仑,对着他摇头摆尾,那姿态让大家想起了琼斯时代的狗模样。

九只狗拥护着拿破仑站在了昔日老少校发表演说的那个突起的台子,然后他宣布,以后取消每周日早晨的碰头会,他解释说,这是完全在浪费时间的会议,根本毫无意义。他会组织一个委员会,当然委员全都是猪,他会亲自统管这个组织,而以后庄园里的大小议题都由这个委员会来决定。猪们会在私下讨论议题,然后再将决策传达给动物们。而星期天动物们还是需要在清晨的时候聚集在一起,要向庄园的旗帜致敬,唱《英格兰兽》,接受下一周的工作任务。生活依旧,唯一不同的是,再也没有了辩论。

雪球的被驱逐给大家带来了不小的震惊,但是拿破仑刚才的一系列决定更加让动物们无比震撼。有一些动物想要抗议,但是没有找到合适的辩词。拳师立起了耳朵,又甩了甩前额,努力地想要理清些头绪,结果依然是迷糊,最后也没说出什么话。倒是几头猪出了声,那是前排的四只小猪,他们不满地尖叫,想要跳起来发言。立刻就见那几只大狗冲着小猪们龇牙咆哮,小猪们顿时没了声音,乖乖地坐了下去。羊们马上开始了咩咩的“四条腿好,两条腿坏”,叫声响亮,持续了一刻钟,于是,讨论的所有机会就在羊们的叫声中彻底失去了。

后来,肩负使命的声响器巡视了一圈庄园后,就新的安排对动物们给予了解释。

“伙伴们,”声响器叫道,“我相信在场的每一个兄弟姐妹都会对拿破仑同志这种勇于担当额外劳动的行为而深感敬佩和感激。当领导不是你们想象的那么简单,享受根本谈不上,每天都有沉甸甸的责任和任务压在肩膀。拿破仑同志也愿意让大家为自己的事情做主,他一直坚信众生平等嘛,但是一旦你们失策了,后果将不堪设想。你们想,要是你们头脑一热相信了雪球的关于风车的鬼话,结局会怎么样?据我们目前所知,说雪球是坏蛋一点也不为过。”

“可是雪球在牛棚战役中表现很英勇啊。”一个动物立马说道。

声响器很快作出了回答:“你认为只有英勇就够了吗?不是的。最重要的是忠诚和服从。牛棚战役中,雪球是有所表现,但是在不久的将来你们就会发现他的作用被过度地夸大了。如今,我们的口号是纪律,铁一般的纪律。我们才刚刚起步,走错一步都是非常危险的,琼斯,我们的敌人就会回来继续奴役我们。你们谁也不希望是这样的结果吧?”

很显然,每次一提到琼斯,动物们都会立刻同意领袖们的意见,他们无论如何也是不愿意让琼斯回来的;如果辩论会导致琼斯回来,那么取消星期天早上的碰头会就必须执行。声响器的那番论证于是又成了不容怀疑的真理了。拳师静静地站在那里思考了很久,然后用一句话表达了他当时的感受,“既然是拿破仑同志说的话,就肯定不会错的”。从此,他的座右铭又多了一句补充,那就是“拿破仑同志永远正确”。

随着天气渐渐暖和起来,春耕开始了。之前作为雪球工作室的那间屋子依旧被封着,至于画在地上的风车的设计图,大家都觉得肯定被擦掉了。他们听从拿破仑的决定,在每个星期日的上午十点,集合在大谷仓里,领取他们这一周的工作任务。老少校的脑壳风干后从果园里挖了出来,被放在旗杆下面的木桩上,和猎枪并排。每周的升旗仪式结束后,动物们按照规定恭恭敬敬走过脑壳,算是瞻仰吧,最后才能进入大谷仓。现在大家的会议座次可不像以前那么随便了,拿破仑进行了新的安排。前台坐着拿破仑和一个叫小不点的猪,小不点在作诗谱曲方面有着惊人的天赋。围在他们外边的是那九只还没成年的狗,呈半圆形坐着。其他的猪坐在后面。剩余的动物面朝他们坐在大谷仓中间。拿破仑的作风一贯是类似军人的粗暴,他给每个动物安排完工作,仅仅唱了一遍《英格兰兽》,就宣布解散了会议。

距离雪球被赶出去已有三个星期了,这个星期天,动物们听到了一个让他们都格外震惊的消息,那就是拿破仑宣布要建造风车。拿破仑在会上并没有解释为什么他会改变主意,只是简单地警告大家,这是额外的工作,并且将会非常艰辛,也许还会缩减他们的食物。原来过去的三周,委员会里的所有猪都在为风车做着准备工作,包括筹划以及一些细微的部分都统统被考虑到了,整个项目被安排得很妥当,他们预计要花两年的时间来建造风车和进行其他项目的改进。

声响器当天晚上就向动物们透露了一个消息,拿破仑从来没有反对过建造风车。事实上,最先提倡建造风车的就是拿破仑。至于雪球,他其实是个剽窃者,那幅画在孵卵室地板上的风车结构图本是拿破仑创作,只不过他画在了自己的笔记上,而雪球把它抄画在了地板上。对于声响器说的那个消息,有的动物提出了疑问:“那为什么拿破仑起初强烈反对风车计划?”声响器圆滑地回答了他,他说,这是拿破仑同志使用的一个计谋,聪明机灵的拿破仑同志很早就发现了雪球有不良的企图,是个危险分子,最早反对修建风车就是为了将雪球赶出去。现在雪球已经被驱赶了,当然计划就能进行了,不会再有干扰了。声响器说,这就是谋略。他一边蹦蹦跳跳,一边晃动着尾巴,嘴巴里又重复了好多遍:“谋略,同志们,谋略啊!”其间还夹杂着欢快的笑声。动物们不大能明白他的话,他讲得如此有说服力,三条狗一直伴在他的身边,配合着他狂叫,令人无比惊恐。动物们原本的疑问就这么搁下了,算是认可了声响器的解释。
