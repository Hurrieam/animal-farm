\chapter{第二章}

在那聚会的三天后,老少校就安详地睡着了,那时正值三月初,他的遗体被埋在果园脚下。

在接下来的三个月之中,这种秘密聚会接二连三举行过几次。动物们一直在琢磨老少校的话,他的预言不知道什么时候能发生,在他们的有生之年是不是还能赶上造反,那种全新的生活观念已经渗进了那些有点头脑的动物中了。所以,他们开始准备,因为那好像是种义务。猪被认为是最聪明的动物,理所当然教育和组织其他动物的工作就落在了他们的身上,而其中两头公猪本是琼斯养着用来卖的,但却是最为优秀和卓越的,名叫雪球和拿破仑。拿破仑是庄园里仅有的一只伯克夏种猪,他长相凶狠、个头庞大,不择手段就是他惯用的手段。相比之下,雪球就没有他那么深的城府,但是很活跃,伶牙俐齿,点子也多。还有一只很显眼的胖猪,脸胖得鼓鼓的,眼珠子骨碌骨碌的,名叫声响器,因为他善于演说,所以在动物中名声很大,别的动物都说他的话啊,能把黑白都颠倒了。他的动作灵巧,声音尖细,尤其在论证某些难以讲解清楚的论点时,他就会一边摆着尾巴,然后边讲解边来回地蹦个不停,这是一个不知为什么就令人信服的绝招。

这三头较为突出的猪认真仔细地研究了老少校的那一番话,把它加以整理,总结成为一套名为动物主义的思想体系。每周几次的秘密会议照常举行(当然是在琼斯先生睡了之后),主要就是向动物们传授这套思想体系。俗话说,万事开头难嘛,那三头猪没少费力气。起初,一些动物还是战战兢兢,改变不了那些传统的思想,他们还将琼斯先生当成主人,觉得对他还存在着可笑的忠诚的义务,同时他们还提出了一些非常幼稚的想法,比如“琼斯先生给我们饭吃,他要是走了,我们会不会饿死”,还有“如果这是我们死后的事,那和我们有什么关系?”或者“如果造反是一定会发生的,那我们关不关心,出不出力不是没有区别嘛”。

三头猪不厌其烦地向动物们解释说他们的那些愚蠢的想法是背离兽主义精神的。但是那头名叫莫丽的白母马着实愚蠢得让人头疼。她问雪球:“造反后,我还有糖吃吗?”

雪球不假思索地说:“那是不可能的,因为庄园里是无法制糖的,再说了,到时候你会拥有更多的燕麦和草料,那时你就会把糖忘得干干净净。”

“那这些丝带我还能继续扎在我的鬃毛上吗?”莫丽又问。

“这是奴隶的象征啊,难道你还想继续被奴役吗?自由是最有价值的,这点你好好思考一下吧。”

雪球盯着莫丽说完后,莫丽犹豫着表示同意,她还是没想明白,她觉得那番说辞貌似并不能让她完全赞同。

乌鸦摩西,那是个令人讨厌的家伙。因为摩西是琼斯喜欢的宠物,他喜欢散布谣言,搬弄是非,还不劳动,这是让动物们生厌的主要原因。摩西声称有一个叫蜜饯山的地方,那里每周七天都是周日,一年都有新鲜的苜蓿,方糖和亚麻子饼就长在树篱上,那么美好而神秘的地方就在天空中云层上面的不远的某处。尽管讨厌摩西,但是一些动物还是相信他的话,认为那是动物们死后的归宿。猪们费尽口舌,他们必须让这些愚蠢的动物明白世界上是不可能有那样的地方的。

拳师和苜蓿无疑是最忠诚的,猪们一直是这么认为的,因为以马的智商想要弄明白任何问题都不太可能,他们是一根筋的动物,既然认定猪是老师,那么凡是猪教的道理,他们就理解吸收,并且还会稍加论证传达给其他动物。大谷仓的集会他俩从不缺席,临近结束时唱的《英格兰兽》也会由他俩起头。

造反就在大家觉得还没准备好的时候开始了,并且一次成功了。话说那段时间琼斯先生时运不佳,因为一场官司赔了钱,郁郁寡欢,拼命酗酒,整天待在厨房,靠在他那把温莎靠椅上懒懒地坐着,除了看报纸,喝酒,就是偶尔喂摩西一点被啤酒泡了的干面包片。琼斯的萎靡连带着他的伙计们也游手好闲,造成了田地长满野草,窝棚顶失修漏雨,篱笆无人修剪,动物们食不果腹。其实之前的琼斯先生虽然刻薄,但是很能干,算得上是一个称职的农场主。

也许琼斯想不到,他的变化让一个新时代有了崛起的机会。那天是六月底的一个星期六,在施洗约翰节(6月24日)的前夕,本是收割牧草的时节,但是琼斯先生却在威灵登的红狮子酒吧喝得烂醉,第二天中午才回家。因为他不在家,他的伙计们根本不会尽心工作,早早挤完奶就跑出去打野兔了,完全忘记了给动物们添加草料。琼斯先生根本没有心情管庄园里的任何事,他一回来就躺在沙发上,把一张《世界新闻报》盖在脸上开始睡觉,直到天黑。动物们饿得实在是忍不住了,一头母牛率先用角顶开了饲料棚的门,于是所有的动物蜂拥地挤进去开始抢食食物。刚醒的琼斯先生本能地和他的四个伙计拿着鞭子赶到饲料棚对动物们进行抽打。动物们忍无可忍了,即便没有事先商量,他们惊人一致地朝虐待他们的人类冲去。他们用角顶,用蹄踢,不一会儿局面就扭转了。琼斯他们哪曾见过这样的阵势,他们一直认为动物们是可以随意鞭打和虐待的,现在他们被吓得六神无主,除了逃跑,他们已经想不到其他办法了。他们五个沿着通往大路的车道拼命地奔跑,后面的动物们乘胜追击。

这庄园的另一个主人,琼斯夫人通过卧室的窗子将刚才的一切尽收眼底,她匆匆忙忙地将一些细软收拾到一个毛毡手提包里,从另一个通道溜出了庄园,乌鸦摩西还是个恋主的家伙,从木架子上蹿起来,拍打着翅膀,呱呱乱叫着追随琼斯夫人。随着关上大门的那“砰”的一声,造反就正式成功了,虽然他们还没有完全回过神来。他们拴上了五道门闩后就意味着,琼斯被驱逐了,曼纳庄园完全由他们说了算了。

他们或许还是不相信自己成功了,直到他们兴奋地沿着庄园的地界奔跑了一圈,确认了在庄园里是真的没有人类存在了。接着,他们马上回到窝棚中,销毁了一切有关人类文明的东西,他们撞开了马厩尽头的那间农具棚,将里面的嚼子、鼻环、狗链,以及用来阉割猪马牛羊的刀子统统扔进了井里。同时,将缰绳、笼头、眼罩还有有损尊严的吊在马嘴下的饲料袋同其他的垃圾一起焚烧了,当然不会少了那让所有动物都感到屈辱的鞭子。火堆噼啪作响,动物们欢呼雀跃。

“饰带,是人类的标志,如同衣服一样,所有的动物都要与人区别开来,我们应该赤身裸体。”说完,雪球就把饰带扔进了火堆。

拳师听完雪球的话,便把他夏天戴的那顶草帽拿出来,这顶草帽本是防止虫蝇钻入耳朵的,他把它和其他一些东西一起扔进了火堆里。

就这样没用多长时间,关于琼斯先生的东西,凡是动物们能想起来的,就全部被销毁了。接下来,动物们回到饲料棚,拿破仑给他们发了双份的饲料,给狗发了双份的饼干。他们在睡前将《英格兰兽》完完整整高歌了七遍,之后安安稳稳地睡了一夜,这样的睡眠是他们从来不曾有过的。

他们习惯性地在黎明时醒来,但转而想起了昨天发生的事情,昨天的壮举立刻使他们兴奋起来,他们全都跑出了窝棚,奔着大牧场就去了,在通往牧场的路上有一个小山包,动物们冲到顶上,在清新的晨曦中注视着大半个庄园,景色一览无余啊。是的,这是他们的——凡是他们目光所能到达的地方都是属于他们的了。这样的念头一发芽,他们就止不住狂喜,他们蹦啊,跳啊,压抑不住极度的兴奋,他们猛地蹿到空中。他们在露珠上打滚,咀嚼着鲜美的牧草,踢起一块块黝黑的田土,使劲嗅着泥块中浓郁的香味。然后,他们逛了整个庄园一周,在无声的赞美中巡视了耕地、牧场、果树园、池塘和树丛,好似是头一次见这些东西般。并且,直到此刻,他们还是无法相信这些完全是他们的了。

后来,他们列队向庄园的窝棚走去,却在琼斯的房子外静静地站住了,这已经不属于琼斯了,是他们的了,但是他们还是惶恐地不敢进去。好一会儿,还是雪球和拿破仑带头用肩撞开了门,动物们才全部涌进了房子,他们小心翼翼地走着,生怕把什么弄乱了。他们踮着蹄子尖一间房一间房地查看,个个缄口结舌,出于一种敬畏,聚精会神地盯着这些难以置信的奢华,盯着镜子、马鬃沙发和那些人类用过的羽绒填充的被褥,还有布鲁塞尔地毯,以及放在客厅里的壁炉架子上的维多利亚女王的石印肖像。他们看得差不多了就走下了楼梯,转身却并没有找到莫丽,返回去楼上,才发现她还在一间最好的卧室里,拿着一条原是琼斯夫人梳妆台上的蓝色丝带,对着镜子傻呵呵地臭美。在大家的严词厉色下,她才不情愿地走了出去。挂在厨房里的火腿都拿去埋了,搁在洗碗室的啤酒桶也让拳师给踢了个洞。除此之外,房间的其他东西倒是没有动。一项决议马上就在这通过了,大家一致同意把这间农场主的住房保存起来作为博物馆,任何动物都不得进去居住。

早餐后,雪球和拿破仑再次把大家召集起来。

“同志们,”雪球说,“现在是六点半,长长的一天才刚刚开始,所以今天我们要开始收割牧草。不过,还有另外一件事要先商量一下。”

这时,大家才知道在过去的三个月,猪们在垃圾堆里翻出了一本琼斯孩子们用过的书,他们从那本书上学习了阅读和书写。拿破仑要来几桶黑漆和白漆,带着大伙来到面向大路的五闩大门。接着,雪球开始了他擅长的活,只见他用蹄子的两个趾夹起了刷子,蘸了白漆将门顶木牌上的“曼纳庄园”涂掉,又蘸了黑漆在那上面写上了“动物庄园”。庄园的名字就此而定了。写完后,大家伙回到窝棚那里,雪球和拿破仑向他们解释道,过去的三个月,经过他们的深思熟虑,他们把动物主义简练地概括为“七诫”,这是动物们必须遵守的法律,庄园里所有的动物都必须遵循它生活,“七诫”就写在大谷仓的墙上。说完,雪球就开始费力地爬之前拿过来的梯子,因为猪在梯子上可不好把握平衡,所以,费了好大劲雪球才爬上去,声响器提着油漆桶站在稍低几格的地方配合,雪球拿着大刷子在涂过柏油的墙上,用无比巨大的三十码外就能看见的白色字母书写“七诫”,内容如下:

七

诫

1.

凡靠两条腿行走者皆为仇敌;

2.

凡靠四肢行走者,或者长翅膀者,皆为朋友;

3.

任何动物不得着衣;

4.

任何动物不得睡床;

5.

任何动物不得饮酒;

6.

任何动物不得杀害其他动物;

7.

所有动物一律平等。

巨大的字迹中透着潇洒,除了把朋友“friend”写成了“freind”,还有其中一个“S”写反了外,其余都是准确无误的。写完后,雪球负责给动物们大声地朗读了一遍,其间动物们频频点头表示完全赞同,较为聪明的一些动物立刻开始背诵了。

“此刻,同志们,”雪球扔下刷子喊道,“到牧场去,我们要争口气,我们要争取比琼斯他们人类更快地收完牧草。”

这时,早就不自在好一会儿的三头母牛开始哞哞地大叫,原来已经有二十四小时没有人挤她们的奶了,她们的奶子快要胀破了。猪稍加思考,就命部下拿来奶桶,成功地给母牛们挤了奶,他们的蹄子干这个活还挺适合的。很快,五个桶就都装满了漂着乳沫的牛奶,很多动物饶有兴趣地看着这些奶桶里的奶。

“怎样处理它们呢?”一个动物提了出来。

一只母鸡接着说:“以前琼斯先生会在我们的饲料中掺一些牛奶的。”

但是站在奶桶边的拿破仑马上打断了大家的谈话,他说:“同志们,我们当务之急是收割牧草,牛奶先放在一边,会被妥善处理的。你们现在马上跟着雪球同志去收割牧草,我随后就到,前进,同志们!牧草等着你们呢!”

于是,响应号召的动物们成群结队地走向牧场,开始了一天的劳动。当他们晚上收割回来后,发现牛奶已经不见了影踪。
