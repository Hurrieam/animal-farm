\chapter{第六章}

接下来的一年里,动物们拼了命地干活,即便像奴隶,流血流汗甚至牺牲,他们都乐在其中。因为他们明白,他们不是为那群压榨他们,奴役他们,无所事事的败类人类效力,而是为了他们自己以及子孙后代的利益而劳作,他们心甘情愿。

自打开春起,动物们每周的工作时间达到了六十个小时,这种情况持续到夏季结束。到了八月,拿破仑提出了新规定,以后星期天的下午也要继续劳动。虽然他把星期天下午的劳动定性为义务劳动,不强迫完全自愿,但是如果谁不参加的话,将被扣除一半的口粮。即便这样,大伙还是发现,有的任务没办法完成。不光收成比去年差了一些,还有两块地由于没有及时耕犁,初夏时就没有种上根茎作物。不难想象,这个冬季会有多么难熬了。

风车工程甚至比想象中还要艰难。按说建造风车的材料算是准备得差不多了,庄园里本就有一处质地很好的石灰矿,加上在一个棚子里找到的大量沙子和水泥。问题出现了:石头体积太大,不符合建造的规格,但是动物们不知道怎么把它打造成尺寸合适的碎块。动物们因为碍于不能直立,所以那些工具他们没办法使用。他们在浪费了几个星期的体力后,总算是想出了一个好主意。原理可以简单地说成是利用地心引力。实践起来是这样的:动物们用绳子绑住大石头,然后集体上阵,牛、马、羊及所有能拽住绳子的动物,有时,猪也会来帮一把,他们慢慢地拉着石头,直到把它拖到矿顶,最后就是奋力把石头推下崖。显然,等石头着地就会碎开。接下来的工作就是运送摔碎的石块,这些体积小的,运送起来要方便得多。装得满满的货车由马来拉,单块的由羊来拖,母山羊穆尔丽和驴子本杰明合伙拉一辆两轮旧车,这个装得要少一些。日日地积累,直到夏天快要过去了,才算是达到了预计的数量。紧接着,猪就开始指挥大家开始了建造工作。

整个采石过程中的艰辛是无法想象的,进度也是出乎意料的慢。动物们要花费整整一天的时间,耗尽身上的每一分气力,才能把一块原石弄到矿顶,但是还会出现懊恼的事,那就是石头从崖上滚落却没有摔碎。拳师成了中流砥柱,他一个的力气是剩下所有动物力气之和,如果没有他,估计很多工作都无法完成。动物们经常会被原石拖着往山下滚,他们除了无助地哭喊别无他法,每每此刻,总是拳师出手奋力拉紧绳索致使石头停止下滑。拳师的拼命是动物们都亲眼所见的,他步履艰难地吃力爬坡,呼吸越来越快,蹄子尖紧紧地抠住地面,汗水完全浸透了他巨大的身躯,动物们对他无不敬佩和称赞。苜蓿时常告诫他要小心身体,不要劳累过度。但是拳师从来不上心,他总是用那两句口头禅来回答任何问题,“我要更加努力工作”,“拿破仑同志永远正确”对他来说足够了。他与那只小公鸡修改了一下协议,把每天清晨提前半小时叫醒他,定为提前三刻钟。即便现在的业余时间被减到很少了,但他仍然把他争取来的这点时间,花在了运送石块上,他独自一个去采石场,没有伙伴帮忙,自己捡碎石,自己装车,并且装得满满的,再独自一个拉到已经定好的风车的建造地点。

这个夏天过得很辛苦,动物们都不停地干活,虽然他们得到的食物不比琼斯给的少,但是也没有多到哪去,整体生活还算过得去。因为不用再供养那五个胡作非为的混蛋人类,只需要养活自己,这相比得来的优势使得他们暂时忘记了他们遇到的挫折和磨难。在干活的很多方面,动物们还是有自己的优势的,能节省劳动力,提高了不少效率。比如锄草这方面的活,动物们就比人类手拿把掐得多了。还有现在动物们团结一致,没有了小偷小摸,牧场和耕地就没必要再隔开了,这样撤掉了栅栏和树篱,动物们就省去了保养这个工作。但是一帆风顺的日子从来就不会太长,到了夏末,一些问题就接踵而来。生活正常物品的短缺越来越不得不被重视,煤油、线绳、狗食饼干还有钉在马掌上的蹄铁,还有种子和人造化肥,以及后来需要的各类工具、建造风车所用的机械设备等等,这些是庄园里无法生产的物品,至于怎么弄到它们,动物们都表示束手无策。

又是一个星期天,早上动物们按照惯例来到大谷仓接受任务。这天,拿破仑宣布了一项新决议:动物庄园以后将要同附近的其他庄园做些交易。他说他这么做并不是有什么商业目的,完全是为了获取那些生活必需品,并且要不惜一切代价拿到建造风车的部件。暂时决定要卖掉的是一垛干草和一部分当年收获的小麦,如果以后还需要钱,就要从鸡蛋下手了,因为在威灵登鸡蛋总是受欢迎的。拿破仑又说了,鸡们应该为此感到骄傲,因为他们将要对风车的建造做出特殊贡献。

动物们这次又感觉到了迷茫。动物主义里是有这几条的:永远不跟人类交往,永远不从事交易,永远不使用钱币,这不是在反抗成功后的第一次会议上就通过了的决议吗?那时候的场景可都没忘呢,甚至可以说是记忆犹新。这时,那四只小猪,就是曾经想反对拿破仑废除碰头会的小猪们又准备发言,同样的,被那几只恶狗的吼叫,把话给生生地咽回去了,继续保持沉默。接着,又是羊们打破了沉默的气氛,他们咩咩地大叫着“四条腿好,两条腿坏”,尴尬的局面就这么被对付过去了。最后,拿破仑抬起前蹄示意大家安静下来,他说他已作好了万全的打算,为了不让动物们中的任何一个去干与人打交道的事,也为了庄园的发展,他要牺牲自己,将所有的压力重担自己挑。他已经联系了一个中介人,他是住在威灵登的一位律师,叫作温普尔,他答应了做动物庄园和外部世界的联络人,会在每周一上午来拜访庄园,捎些信息。不例外的,拿破仑最后是在动物庄园万岁的呼喊声中结束讲话的。同样不例外的,动物们在唱完一遍《英格兰兽》后就匆匆离场了。

声响器又在这个新决议提出后,来到动物当中说出一些话来使他们安心。他言之凿凿地保证,根本就没有“不准从事交易和使用钱币”这样的决议,更何谈已经通过了呢?之所以会有疑问,那就是一种猜测嘛,凭空想象的,如果追根究底,可能是当初雪球那个坏家伙的一个谎言吧。有些动物还是半信半疑的,狡猾的声响器就问他们:“你们敢确定这不是你们梦里的事情?你们有这些决议的记录吗?记载在哪了呢?”当然,这些是从来没被记录的。动物们因此也就觉得是自己弄错了。

就如决议里定的,温普尔在每周一都要来庄园。这个人长着络腮胡子,身材矮小,但是丝毫不影响他的外表透出狡诈的气息。他经办的范围很小,但他人精明得过分,他早就意识到了动物庄园会需要中介人,并且作为头一人,佣金将非常可观。他在动物们面前走来走去,动物们对他还是畏惧的,所以尽量躲得远一点。不过,在这些四条腿的动物看到他们的领袖拿破仑同志用两条腿站立着向温普尔发号施令,他们的心里还是没来由地骄傲了一下,之前对新决议的抵触情绪也得到了很大程度的缓解。目前动物们和人类的关系已经今非昔比了,但是在人类那一方,动物庄园多一分繁荣,他们就对动物们多一分憎恶,甚至于到了仇视的最大值。每一个人类都坚持一个信条:动物庄园的下场就是破产,那个建造风车的项目毫无疑问是会失败的。这些消息大都从酒馆里传出来,因为人类常常聚在酒馆里谈论动物庄园的事情,用图表论证了风车倒塌的必然性,或者是建起来后的风车也不会运转等等带有嘲笑意味的谈话。但是出现了一个比较有意思的迹象,那就是在谈话中,人类对于动物们自己把庄园管理得比较井井有条,而或多或少地对他们刮目相看,主要表现就是,人类在提到庄园的名称时,已经默认了动物庄园,而不再说是曼纳庄园。而琼斯也基本已经不再受到其他人的支持,他自己成天萎靡不振,也放弃了反击,最后干脆搬到了国内的另外一个地方生活去了。动物庄园能和外界联系,还真是多亏了这个温普尔,但是出现了小道消息,还不断升级,内容是拿破仑还会选择跟两个邻居庄园的农场主——福科斯伍德的皮尔金顿先生或平彻菲尔德的弗雷德里克先生,签订一个商业性质的合同,不过需要强调的是,不会同时跟两家签合同。

大概就在那个节骨眼上,猪们突然入住了庄园主的房子,也就是琼斯以前住的地方。动物们再次感到很不舒服,他们记得之前是有通过这样一条禁止住人类房子的决议。可是声响器再次让他们认识到自己思想的不准确性。他说,作为首领、领袖(近来,他在提到拿破仑时开始用领袖称谓),找个安静的地方,方能好好休息,况且身份比以前尊贵,理应住到比猪圈更能符合身份的房屋里。还让动物们充满疑问的事情是猪们不但在厨房里进食还在客厅里娱乐,最后竟然睡在床上。拳师倒是没有什么表示,他只说了他的经典句子“拿破仑同志永远正确”。苜蓿就纳闷了,因为她记得有一条规定不许睡床的决议啊,她不甘心地跑到大谷仓尽头盯着墙上的七诫,反复地看,想要从中找到答案,但是遗憾的是,她不识字,连个字母都忘了。她只好找来山羊穆尔丽帮忙。

“穆尔丽,”她说,“你帮我把《七诫》中的第四条读一下吧,是不是有永远不许睡在床上的字眼?”

穆尔丽聚精会神地盯着墙上看了半天,费劲地拼出来。

她是这么给苜蓿念的:“墙上写的是‘任何动物都不准睡床铺被单’。”

苜蓿更加迷茫了,她完全不记得戒律里有提到被单,但是它怎么会写在墙上的?那么原来就有它吧。此刻,身旁围着两三条狗的声响器赶巧来到这里,他总是能从不一般的角度来对问题进行剖析。

“看来,同志们,你们已经知道我们猪在庄主床上睡觉的事了?这有什么问题吗?你们不是又找出一条反对在床上睡觉的戒律吧?床的功能只是睡觉,没有那么多的意义。能睡觉的地方都可以叫作床,比如窝棚里的那堆干草。为什么这条戒律针对被单呢?因为被单是人类的产物。我们只是睡在床上,覆盖在床上的被单早就被我们全都撤掉了,我们睡在毯子里。这也是舒适的床,床的形式太多了。同志们,我告诉你们,我们付出了大量的脑力,所有的脑力工作都是我们干的,所以现在的舒适程度远远跟我们的付出不成正比。同志们,你们不会残忍到连我们的休息权都剥夺吧?你们不会想让我们过于劳累而失职吧?你们不会想让琼斯趁着我们失职的时候回来吧?”

一提到琼斯会趁机回来,动物们立刻就消除了声响器的疑虑。他们从此后再也没有谈论任何有关猪睡在庄主床上的事了。没过几天,一项新决议从猪委员会里诞生了,那就是以后每天早上猪可以晚起一个小时,其他动物正常,对此,大家都没有什么抱怨。

一直到秋天,动物们都在累得要死,却还算快乐的矛盾中徘徊。说起来也已经一年了,这一年他们不停地辛苦干活,但是准备过冬的饲料依旧不宽裕,因为他们卖掉了部分的干草和谷物。如今风车工程差不多建了一半了,算是可以补偿这一切。动物们在秋收后,趁着天气好,持续晴朗无雨,铆足了劲地干活,比以前还要卖力。他们整天不辞辛劳地拖着石块,来回奔波。他们通过加倍的劳作使墙在一天之内可以增高一英尺,对此,他们无比欣慰,认为所有的付出都是值得的。拳师甚至在夜间也会时常出来,借着月光干上一两个小时。动物们喜欢在工作空闲时在建了一半的风车外绕来绕去,赞叹着墙体的坚固和笔直,其实就是在为自己能建造这样规模宏大的工程感到惊喜。老本杰明在风车的事情上一直保持着冷淡的情绪,他依旧像以前那样,说了句驴子的寿命都很长这样神秘的话后,就保持沉默了。

十一月夹杂着西南风来到了。随之而来的就是停工,因为湿度太大了,水泥没法搅拌。天气越来越恶劣,终于在一个夜晚达到上限,那晚狂风大作,吹得庄园里的窝棚连地基都开始摇晃了,还卷走了大谷仓顶棚的瓦片。鸡群在睡梦中被一声枪响给惊醒,他们恐惧得只知道咯咯乱叫。第二天一早,从窝棚里走出来的动物们就发现庄园里一片狼藉,旗杆被吹倒横在地上,果园里的大榆树被连根拔起,更让他们不可思议的是,风车被摧毁了,成了一堆碎片,他们看到了这一幕可怕的场景,抑制不住发出了绝望的号叫。

动物们在没有号召的情况下集体冲向风车,很少外出的拿破仑率先跑到了队伍的最前面。在那里,他们的奋斗成果倒了,他们千方百计不辞辛劳弄碎拉到这里又砌上的石头,此刻四下里散落着。动物们沮丧极了,他们盯着倒塌下来的碎石块,表示默哀。拿破仑一句话也没说,他来回检查着,不时闻闻地面上的味道,尾巴开始变得坚硬,还会伴随着剧烈的来回抽动,他这样的表现说明他现在的心理活动高度紧张。突然,他好像想到了什么办法,就停在那儿不动了。

然后,他平静地说:“你们知道这是谁干的吗?你们想过会是谁半夜潜进来毁掉了我们的风车吗?是雪球!”他的声调突然提高了八度用近乎咆哮的语气吼道,“这是雪球干的坏事!这个用心险恶的叛徒,他早就计划好了,他摸黑钻进来偷偷跑到这里,就是为了来摧毁我们这一年的劳动成果。他就是要阻挠我们的计划,他报复我们当年驱逐他,真是太可耻了!所以我宣布,要将雪球处死。谁要是能把他就地正法,就颁发给谁‘二级动物英雄’勋章,还有十八升苹果的物质奖励。谁要是能活捉他,就能得到三十六升的苹果。”

动物们被拿破仑的一番话惊得不知所以,他们对于雪球会犯下如此深重的罪孽感到十分愤慨,他们在互相发泄了一通之后,开始思考怎么在雪球再次来袭时抓住他。基本同步的是,动物们在离小山包不远的草地上,发现了几只脚印,仅仅几码远就消失了踪迹,但很显然,那是一只猪留下来的,他似乎是去了树篱上的一个缺口。拿破仑对脚印仔细地嗅,得出了结论:脚印就是雪球留下的。按照他的思考得来的结论,雪球很可能是从福科斯伍德庄园那边来的。

拿破仑在判断完蹄印后,就对着动物们大声地说道:“同志们,伙伴们,我们不能再犹豫了,不能再迟疑了,我们还要去干很多的工作,就从今天开始,从现在开始,我们重整旗鼓,不论晴天还是雨天,不论天寒还是地冻,我们要把满腔的热情投入到重新建造风车的工作中去。我们要让叛徒雪球见识一下我们的战斗力,我们不能让他得逞,不能让他易如反掌地把我们的伟大成果给毁了。伙伴们,牢记我们的计划,不能因为雪球的破坏就使它有任何的变动,相反我们一定会如期完成的。同志们,前进!风车万岁!动物庄园万岁!”
