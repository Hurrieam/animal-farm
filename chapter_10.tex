\chapter{第十章}

恍惚间几个年头就过去了。时光流逝,岁月穿梭,现如今,还能记住造反前的那些旧日子的只剩下苜蓿、本杰明、乌鸦摩西和一些猪,那些寿命短的动物早已经被岁月带走了。

我们熟识的那些动物中,穆尔丽死了,蓝铃铛、杰西还有钳子都死了。就连琼斯也死在了别的地方,据说是国内的另外一个地方的一个酒鬼家中。雪球再也没被提起,拳师也被遗忘了,现在只有零星几个曾经与他们交好的动物还会记得他们。如今的苜蓿到了老年,身材胖得走了样,关节僵硬,一团眼屎总是挂在眼角。其实,她都超过退休年龄两岁了,但是依然还在工作,事实是,没有任何一个动物真正退休。关于那个留给退休动物大牧场的一角以供休养的议题早就不被提起。拿破仑现在已经达到了三百多磅,是一头成年的公猪了。声响器的眼睛被脸上的赘肉挤得快要睁不开了。只有老本杰明,似乎没什么变化,只是他鼻子和嘴旁边的毛比以前灰了点,他在拳师死后,更加不喜言语,对事情更加漠不关心。

虽然涨幅没有当时预算的那么大,但是庄园里现在的畜生数量还是比以前多了。现在造反对于后来出生的这些年轻的动物来说,就像一个传说,只是口口相传,模糊不清;至于那些买进的动物,他们来到这个庄园前,压根就没有听说过造反的事。庄园里现在除了苜蓿外,还有三匹马,只是这三匹马太愚蠢,没有一个能认识字母表中B以后的字母,但是就劳动来说,他们还是不错的,他们的身姿健壮挺拔,温顺又积极能干,是大家普遍赞许的好同志。他们对苜蓿格外敬重,近乎是那种孝顺的方式。对于造反和动物主义的说法,只要是他们能够得到的信息,他们几乎全盘接受,特别是出自苜蓿之口的。但是有一个问题,对于这些信息,他们是否能够理解,就不得而知了。

综观现在的动物庄园,一派繁荣的景象,更为有效的组织也在某些动物的谋划下建立起来了。动物们又从皮尔金顿先生手里买来了两块地,扩大了自己的领土范围,风车也终于在大家的努力和期盼下建造起来了,但是却并不是用来发电的,而是被用来碾磨谷物,为庄园带来了不少的收入。他们还购买了属于他们自己的一台脱粒机和一台草料升降机,此外他们也新建了不少各种各样的建筑物。就连温普尔也为他自己置办了一辆轻便的双轮马车,以方便他来庄园。动物们现在正在致力于建造一架新的风车,据说这架风车会被用来发电。但是雪球当年曾经提及的那种奢华的生活——有电灯和具备冷热水的窝棚,每周只用工作三天,如今不会再被提起,因为拿破仑曾经批评过这样的想法,说它与动物主义精神是相互抵触的,动物主义倡导纯粹的幸福,而纯粹的幸福是要从勤奋的工作和简朴的生活中获得的。

整个庄园看似已经很富裕了,但是不知是什么原因,在动物们身上仿佛一点也体现不出来,当然这里是不包括猪和狗的。也许其中会有一小点原因是因为有那么多的猪和狗吧。动物们富不起来,并不是他们不劳动,相反,他们总是没完没了地干活,以他们自己的方式劳动,就如声响器反反复复说的那样,他在庄园的监督和组织方面有数不尽的活要干。他说的那些工作,很多的动物都理解不了,他解释说是因为很多动物太过无知了。比如,声响器说,猪们每天会花大把的精力放在处理那些所谓的“档案”“报告”“会议记录”还有“备忘录”诸如此类的神秘事务上。那些大片大片的纸张,上面必须写上满满的字,等到写满了字以后,又得把它们扔进火炉里烧掉。声响器一直强调他们所做的那些对庄园的幸福安康起到至关重要的作用。其实直到现今,无论是猪还是狗,都没有在生产劳动中到前线贡献出一点力量,生产一粒粮食;但他们的数量众多,胃口也出奇地好。

其他的动物,他们的生活在他们看来还是没有变化,经常忍饥挨饿,还是睡在草里,喝着池塘里的水,在田地里忙碌劳作;夏天要忍受蚊蝇的侵袭,到了冬天也不能幸免于寒冷的侵袭。他们当中那些年老的动物会经常在回忆中搜索关于早期造反的那些日子,就是琼斯刚被驱逐的那段时间,那时的情况和现在相比,是好,还是糟呢?但是时间过去得太久了,他们怎么也想不起来了。现在的生活水平没有什么可以拿来参照的,只有声响器时不时地给他们报告一些数字,通过这些数字,声响器一直不变地告诉他们,他们的生活是越来越好的,肯定是越来越好的。动物们对于这些表现得难以理解,但是大多时候,他们没有时间仔细揣摩这些事情,活儿太多了。只有老本杰明说他能记起过往发生的所有事情,他还说他知道过去的那些事物不会有多好,现在的也不会,将来更不会有多好,但也不会差到哪去。他告诉大家说,饥饿、痛苦、磨难、失望是现实生活中无法改变的法则。

动物们依然有他们自己的念想,更准确地说,他们依然有那种作为动物庄园里一员的那种自身优越感,这种荣誉让他们为之骄傲自豪。因为他们的庄园在英格兰这个国家里是独一无二的,是由他们动物自己来管理的,对于这一点,所有的动物,连最年轻的,还有那些从外面买进来的,来自十英里甚至二十英里以外的新成员,都沾沾自喜,钦佩有加。猎枪的声音一旦响起,那种自豪就荡漾在他们的心中,再看到象征动物主义的绿旗在旗杆上飘扬时,那种激动的感觉就难以言表了。此时,他们一定会把话题转向往昔的辉煌岁月,经过时间的沉淀,或许我们该称为史诗。从起初的琼斯遭到驱逐,再到“七诫”的镌刻,然后是那几场与人类的胜利作战。他们的信仰一直在,没有改变过,他们的那些梦想从来一个都没有忘记过,当年老少校预言的“动物共和国”,以及那时人类的足迹将消失在所有的英格兰的绿色原野上,这些一直都是他们不曾忘却的,他们相信这些预言终将成为现实,也许在不久的将来,也许现在的动物们有生之年都等不来这一天了,但是它一定会实现的。现在庄园里的每一个动物都知道它是一个事实,《英格兰兽》这支曲子已经又在动物们中悄悄传唱开了,虽然是小声地哼唱,偷偷地,不敢高歌,但是这是个事实。他们现在的生活仍旧困苦,并且他们抱有的希望依旧没有全部实现,但是有个信念在他们心中支撑着他们,他们知道他们跟别的动物是不一样的,虽然没有足够的食物来填饱他们的肚子,但是那也不是因为食物进了人类的肚子;虽然还要辛苦地劳动,但是这些劳动是为了他们自己,并不是人类。他们都是四条腿的动物,没有主仆之分,他们所有的动物都是平等的。

初夏的一天,声响器把绵羊群带了出去,他领着他们来到庄园的另一头,那里是一片荒地,地里遍布着新鲜的桦树苗。声响器在一边监督,绵羊们在那里足足啃了一整天的树叶。到了傍晚,声响器对他们说,天气暖和了,就可以留在那里过夜了。就这样,声响器自己回到了庄园里。他对其他的动物解释说,绵羊们正在学习一首新歌,需要他秘密教授。于是,整整一个星期,绵羊们都不见一丝踪影,声响器在白天也是不照面的。

在一个愉快的傍晚,绵羊们回到了庄园。那时,动物们刚刚结束了一天辛苦的劳作往庄园里走,突然一声透着惊恐的马的嘶鸣在院子中响起,动物们吓得止住了脚步,有的动物听出那是苜蓿的声音,又一声嘶鸣响起,动物们立刻焦急地往院子里狂奔过去,然后,他们也目睹了苜蓿眼见的那一幕:

一头猪正在用他的两条后腿走路。

是的,那是声响器。他似乎还没有习惯用两条腿支撑他那肥硕的身体,虽然看起来很笨拙,但还是能保持着平衡的,此刻他正溜达着要穿过院子。不大一会儿,排成一长队的猪从庄园主院子门口走过,他们都是用两条后腿在走路,其中有的猪走得还不错,有的走得还不够稳当,看起来他们应该需要一根棍子支撑着。不过,他们都算成功地绕着院子走了一圈。伴着那只小黑公鸡尖细的啼叫声和那群狗可怕的吠声,拿破仑同志粉墨登场,他的那些贴身侍卫又蹿又跳地围在他周围,他则挺直着身子,带着一种强大威风的气场,用那饱含傲慢的小眼神在全场中扫视了一圈。

一根鞭子被夹在他的蹄趾中。

此刻的院子气压极低,动物们吓得大气不敢喘一下,四周静悄悄的,惊讶夹杂着恐惧弥漫在动物们中间,他们下意识地挤在一起,看着猪们排着长长的队伍在院子中缓慢地行走。现在的情形仿佛是世界已经被颠倒了。动物们在刚刚经历了惊讶—恐惧—困惑这一系列的心理活动后,稍稍地缓过神来,然后他们顾不得那么多,就要开始发出抗议,要知道在以前,他们会顾及对狗的恐惧,还有他们长年累月养成的不抱怨任何事、从不批评任何做法的习惯,但是现在,他们真的要不顾一切了。然而,绵羊们仿佛就在此时接到了信号似的,赶在动物们抗议之前,猛然爆发了气势汹汹的叫喊声:

“四条腿好,两条腿更好!四条腿好,两条腿更好!四条腿好,两条腿更好!”

这叫声冲天的咩咩声持续了长达五分多钟,等到绵羊们安静下来后,那些排成长队的猪早已经走回了庄园主的房子里了,表示抗议的机会也就随之流产了。

苜蓿用她的鼻子碰了碰本杰明的肩膀,本杰明感觉到后,回过头去,他看见苜蓿的本就昏花的双眼比以前更加灰暗,苜蓿轻轻拽着他的鬃毛,把他拉到了写着“七诫”的大谷仓的那一头,一路上没有言语。他们面对着涂过柏油的墙壁站着,静静地凝视着墙上用白漆写下的字。

一两分钟过后,还是苜蓿先开了口:“我的眼睛越来越昏花了,虽然我以前也认不全这墙上的字,但现在我还是觉得这上面的字跟以前不一样了,本杰明,你看看这墙上的‘七诫’和以前一样吗?”

这一次,本杰明没有拒绝,和以往真不一样,他把墙上的字念给苜蓿听,如今的那面墙上只剩下一条戒律,其他的全不见了。那条戒律是这样写的:

凡动物一律平等

但有些动物比其他动物更加平等

第二天,那些负责庄园监督工作的猪,个个的蹄子上都夹着一根鞭子,由于明白了那条戒律,动物们在面对猪们的做法时也就不会再大惊小怪。听说,猪们买了一台无线电供他们自己使用,并开始着手准备安装一部电话,此外,对于他们还订阅了《约翰牛报》《花边新闻报》和《每日镜报》,动物们也不觉得奇怪;看到拿破仑嘴里叼着烟斗,在庄园主院子里的花园里漫步,动物们也不稀奇。还有,猪们穿着从庄主屋子里的衣柜里翻出来的琼斯先生的衣服;拿破仑也公然将一件黑上衣和一条猎装裤套在了自己身上,还绑上了皮绑腿;一头深得他宠爱的母猪也穿上了曾经琼斯太太仅在星期天才舍得穿上的一件波纹绸连衣裙。面对如今猪们种种的行为,动物们都选择了接受。

很多轻便的双轮马车在一周后的某个下午驶进了庄园的大门。原来这是一个应邀来参观考察的代表团,成员是邻近的一些庄园主。他们被带领着参观了庄园的每一个地方。他们对所参观到的这些地方都给予了极高的评价,尤其是风车。动物们一直都在萝卜地里拔草,他们尽量把头低下去,几乎碰到了地面,根本无法猜测他们这么做是害怕猪呢,还是更害怕那些来参观的人,但是,这些不影响他们认真而卖力地工作。

到了晚上,一阵阵的欢声笑语和热闹的唱歌声从庄园主的院子里传出来。动物们也被这突然来到的混杂的声音所吸引,这是人类和动物首次在和平的关系下聚在一起,究竟会发生什么事情呢?好奇心驱使他们不约而同地朝着庄园主院子的花园溜去,并且是以极轻的脚步走过去的。

一直走到门口,或许是因为害怕,他们都停下了脚步,苜蓿大着胆子带头走进去,其他的动物在后面小心地跟着,他们踮着蹄尖走到了房子边上,一些个头大点的动物就能透过餐厅窗户看见屋子里面的情形。那里面,六个庄园主和六头地位显赫的猪围坐在一张长桌边,那个桌子顶端代表东道主的位子上毫无疑问坐着的是拿破仑。猪们以一副相当悠闲自在的姿态坐在椅子上。他们应该是饶有兴趣地在玩牌类的游戏,中间偶尔会暂停,显然是为了干杯,他们之间在不停地传着一把大酒壶,酒壶里盛着啤酒,然后啤酒又被不断地斟满每个杯子。他们太过于专注,以至于根本没有发现窗外一张张惊愕的面孔。

首先举着杯子站起来的是福科斯伍德庄园的皮尔金顿先生。他说,此时此刻,他有必要说几句话,然后再跟在座的诸位干上一杯。

他说,他现在被一种巨大的欢喜之感充斥着胸中,他相信,其他的诸位现在肯定也是同样的感觉。

长期以来的种种猜疑和误解现在都已经结束了。曾经的一个时期,他们这些作为尊敬的动物庄园所有者的人类邻居,当然包括他自己还有在座的任何一位,都不会认同那些一定程度上的猜忌,他们不愿意承认那是敌视。屡次发生过不幸的事件,也曾流行过错误的观念。他们觉得一个归猪管理的庄园,多多少少都会不正常,还很有可能会给周边的庄园带来不安全的因素。所以,为数众多的庄园主在还没有进行充分的考证前就肆意下了定论:这样的庄园,里面肯定是恣意妄为的歪风邪气。这样的状况也肯定会影响到自己庄园里的动物们甚至影响到雇员,使他们惶惶不安。但是,在今天他和他的朋友们参观完这个庄园之后,所有的怀疑都荡然无存,他们亲眼考察了庄园里的每一个角落,他们有了巨大的发现,这些发现是:这里的操作方法是最现代的,并且纪律很严明,大家井然有序地进行自己的工作,这在其他的庄园看来一定是楷模。还有一点他认为是完全正确的,那就是动物庄园里的下层动物们吃的跟全国的动物来比是最少的,而干的活却比全国任何动物都要多得多。很显然,在看到了庄园里的这些情况后,他们打算尽快地把这些富有特色的东西引进到他们自己的庄园里去。

在他准备结束他的讲话之前,他再次重申了他们要继续将这份曾经存在于动物庄园和其他相邻庄园之间的友好感情维持下去。没有一种利害关系存在于猪和人之间,以前不曾有过,以后就更不可能有。他们所面临的困难和奋斗的目标都是一样的,还有劳工问题,这在任何地方难道不都是一样的吗?讲到这里,皮尔金顿先生显然是想说出一句俏皮话,因为他自己都忍俊不禁致使他强忍了一会儿,呈多层状的下巴都憋成紫色的了,终于这句精心准备的话被他说出来了,他说:“如果你们有你们的下层动物需要对付,我们同样有我们的下层阶级。”这是一句名言警句啊,在座的各位被这句话引得哄堂大笑。之后,对于在庄园看见的饲料供应量少,工作时间长,而动物们普遍没有松懈疲沓的现状,皮尔金顿先生再次对猪们表示祝贺。

最后,皮尔金顿先生说到,请各位把各自的酒杯斟得满满的,然后集体起立,“先生们”,他说,“我最后敬你们一杯,先生们,为动物庄园的繁荣兴旺干杯!”

说完,爆发了一片热烈的喝彩声和跺脚声。拿破仑同志是最兴奋的,他激动地离开了自己的座位,绕到桌子的另一边——皮尔金顿先生面前,和他碰了杯,然后一饮而尽。之后,完全靠双脚站立的拿破仑同志在等到喝彩声平静下来后,表示他也要说上几句。

他如之前的每次演讲一样,开门见山,直奔主题,并且语句简明扼要。他说,对于那个对动物庄园误解的时代的结束,他也感到非常开心,至于那些长时间流传的谣言,他有证据显示是一些居心不良的仇敌散播的,那些谣言是不可信的,说在他自己和同事们的观念中,有某种东西是带颠覆性甚至革命性的。还有他们被误解为企图煽动周围庄园的动物一起来造反。但是,真相总会大白的,谎言是遮盖不了真相的。其实,他们唯一的愿望就是与他们的所有邻居和平共处,保持正常的贸易,这个愿望从没变过,过去是,现在是,未来仍是。他还补充到,现在这家庄园是合作性质的企业,他有幸在这里监管,那张地契由他亲自保管,但是归所有的猪共有。

他还说,任何的旧的猜忌都会消失,他相信不会残留的。为了更进一步地增加彼此的信任,在前不久,他们还专门对庄园里的规章制度作了一些修订。庄园里还有一个长期养成的愚蠢的习惯,那就是动物们之间互称“同志”,这个要被禁止了。此外还有一个陋习,很奇怪,但是已经无从考证它的由来了,那就是每个星期天早上必须列队向一个头盖骨致敬,这个头盖骨之前钉放在花园里的一个木桩上,当然,这个也要取消。头盖骨已经被埋掉了。这些参观者如果有留意到旗杆顶上迎风飘扬的绿旗,他们应该就会发现原先旗子上标记的白色蹄掌和犄角已经不见了,今后整个旗子就是全绿色的。

他说,刚才皮尔金顿先生的演讲很精彩也很亲切,但有一处错误需要指出,当然这不怪皮尔金顿先生,因为他不可能知道,他之前一直提到的“动物庄园”,从今后就不会再说,拿破仑宣布,并且是第一次宣布,今后,庄园将被称为“曼纳庄园”——他相信,这应该才是庄园的真正名字。

最后,他对他的发言作了总结:“先生们,我将给你们同以前一样的祝词,但是这次的方式是不同的,你们先把酒杯斟满。先生们,我的祝词是:为曼纳庄园的繁荣兴旺干杯!”

同刚才一样,拿破仑的话音刚落,又一阵衷心而热烈的喝彩声响起,酒被他们一饮而尽。此刻,我们看看站在外面的动物们,他们凝视着屋里的这一幅情景时,他们似乎觉得发生了什么怪事。好像是猪的脸孔起了什么变化吗?苜蓿用她那昏花的双眼从一张面孔掠到另一张面孔。他们那些家伙中,有的有着五个下巴,有四个的,还有三个的。但是再继续观察,还是觉得有什么东西正在逐渐消失和发生变化,但是究竟是什么?这时,屋里的喝彩声渐渐低到没了,他们又开始围着桌子,拿着扑克牌,继续刚才被打断的游戏,外面的动物们悄无声息地离开了。

但是他们还没走出二十码的路程,就被庄园主院子里传来的大吵大闹吓得猛地停住了脚步。动物们再一次回到院子里,透过窗子向里面张望。没错,屋子里的确是在争吵,而且很激烈,既有大吵大闹的,还有愤怒拍桌子的,有犀利的满是怀疑的眼光,也有矢口否认的。造成争吵的原因好像是拿破仑同皮尔金顿同时打出了一张黑桃A。

十二个嗓门一齐怒气冲天地吼叫,他们是多么相似。现在,终于明白了猪的面孔上发生了什么变化。窗外的动物们还在向里面张望,他们的目光从人转到猪,又从猪转到人,再从人转到猪;现在,已经不可能分得清哪张脸是猪的,哪张脸是人的了。
