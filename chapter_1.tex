\chapter{第一章}

故事发生在曼纳庄园的一个平常夜晚里。庄园的主人琼斯先生,拖着醉醺醺的身体,只锁好了鸡棚,却忘记了关好那些里面的小门。他颤颤巍巍地提着摇曳的马灯,穿过漆黑的庭院。走进后门,机械地甩掉了臭皮靴,拿起洗碗间的酒桶里的最后一杯啤酒,一饮而尽。这才终于摸进卧房,倒在了鼾声雷动的琼斯夫人身旁。

当卧室陷入黑暗的一瞬间,立刻有一阵扑棱棱的声音响彻在庄园的上空。在白天里,这件事就在庄园里传得沸沸扬扬,就是那头获得“中级白鬃毛”奖荣誉的公猪老少校,决定把自己前一天夜里做的一个奇异的梦,讲给其他的动物听。

大家商量着,当琼斯先生把自己灌醉绝对不会再来谷仓的时候,就立刻集合所有的动物到大谷仓内。老少校(虽然当年他获奖的时候用的是“威灵顿帅哥”这个名字,但是大家习惯这么叫他)一直是庄园里最受尊敬的动物,所以能够听他讲事情,即使失去一个小时的睡眠时间,大家还是满心欢喜的。

少校早已安坐在大谷仓隆起的台子上,在他头顶上方,一盏马灯从房梁上垂下,身底的垫子下铺着厚厚的干草。他已经十二岁了,虽然最近微微发福,但依然是一只仪表英俊的猪。不要去在意他从来没有长出犬牙来,他依然呈现出一副慈爱聪慧的智者模样。很快,其他的动物陆陆续续地走进谷仓,根据自己不同的喜好坐下来。首先是三条分别叫作“蓝铃铛”“杰西”和“钳子”的狗进场,马上台子前面的稻草上就被几头猪给占领了。窗台上母鸡们咕咕叫着,房梁上鸽子们扑腾着翅膀。羊和牛悄悄地躺在猪的身后,开始了今天的反刍。拳师和苜蓿是拉两轮货车的两匹马,他们并肩走进来,茸毛包裹的马蹄总是轻轻稳稳地落下每一步,好像生怕踩到藏匿在干草堆里的一些小动物。苜蓿是一只中年母马,健壮而又温婉,在经历过四次生育之后,就再也无法恢复往昔的体态形貌。拳师则是高大威猛,将近六英尺(1英尺=0.3048米,后同)高的个头,结实得赛过两匹普通的马。但是,那道从鼻梁生出来的白毛,给他的威猛打了折扣,看上去有少许傻气。他的确没有异于常人的高智商,但是他执着坚强的性格和吃苦耐劳的工作表现,获得了庄园里的普遍赞誉。紧接着,山羊穆尔丽和被称为本杰明的驴子也出现了。本杰明是庄园里年岁最大的动物,性格乖戾,总是缄默不语,一旦他要发表言论,总是说些不温不火的奇谈怪论。例如,他总说尽管上帝赐给他尾巴是为了驱赶该死的苍蝇,但是他宁愿两者都不要。他是庄园里最严肃、最不知道分享欢乐的动物。一旦有人追究他冷漠的原因,他会一本正经地说没有什么新奇引起激动的事情。但是,大家都能看得出来,唯一令他佩服和尊敬的就是拳师。你总会在星期天,看到他们两个在果园远处的一片牧场上,默默不语地并肩吃着青草。

让我们言归正传。当拳师和苜蓿刚刚趴下的时候,一群没有妈妈的小鸭子呼呼啦啦地进了大谷仓,一边无力地叫,一边左看右看地想要找一个不被踩踏的地方。很快他们发现了一个好地方,那就是苜蓿那犹如一堵墙一样的前腿所围成的一小块安身之所,于是他们就依偎在苜蓿给他们制造的这一小块天地里进入了梦乡。就在会议快要开始时又进来一匹白母马,虽然漂亮但是却很愚蠢,她就是给琼斯先生拉双轮轻便车的莫丽。只见她嘴里嚼着一块方糖,一摇一摆娇滴滴地走进来,找了个靠前的位置,刚站好就开始试图通过抖动她白色的鬃毛把大家的眼光吸引到那些扎在鬃毛上的红丝带上。最后一个来的是猫,她硬是挤进了拳师和苜蓿的中间,她对这个热乎的地方很是满意,因为在少校演说时她根本一个字都没听进去,只在那里兀自发出咕噜咕噜的声音。

少校看了看,发现除了那只驯化了的乌鸦摩西(此刻他正在庄主院后门背后的横木架上睡觉),所有的动物都到场了,并且都全神贯注地等着他发言,于是他清了清嗓子开始说:

“同志们,我想我昨晚做了个怪梦的事情你们已经都听说了。但是,我们暂且不说那个梦,让我给你们讲点别的。同志们,很遗憾,恐怕我的时日不长了。在我去世之前,我想把我所获得的智慧传于你们,我认为这是我自己应该尽的义务。我想要给你们讲的是什么呢?我能够说的是什么呢?是我用我漫长的一生,花了大量的躺在圈里的时间去沉思,最终参悟出一个哲理——那就是活在世上本应该是什么样的。

“同志们,我们应该正视这个问题,直到如今我们是怎么生活的?让我们看看这个显而易见的事实:从我们出生开始,仅仅能得到勉强维持生活的食物,吃不饱还要被强迫着去干活,直到最后把气力消耗殆尽。当我们一失去使用价值,我们就会被残忍地杀掉。在英格兰,没有动物是有自由的,我们在满周岁后何曾享受过幸福或休闲?我们动物的这一生是多么凄惨、短促和备受奴役啊!

“难道我们这凄惨的一生是上天注定的吗?难道所有居住在这里的我们的兄弟姐妹不能过上体面的生活是因为英格兰是穷山恶水的地方吗?不!完全不!我们这里是物产丰富的地方,有肥沃的地,宜人的气候,能够养活比现在要多得多的生物。单单我们生活的这个庄园,就足以将十二匹马、二十头牛和数百只羊养活得令我们无法想象的舒适和体面呢。然而看看我们现在为什么却过着这种惨兮兮的日子?同志们,你们想过罪魁祸首是谁吗?是人啊!他们把我们所有的劳动成果窃取了,是他们导致了我们的饥饿和过度劳累,人才是我们过不上好日子的真正仇敌。

“在所有的生灵当中,只有人类不劳而获,他们没有产奶的技能,也没有下蛋的本事,拉不动犁吧,跑得比兔子还慢。然而,那些没用的人类却是所有动物的主人,他榨取我们的绝大部分劳动所得,回报我们的却仅仅是能填饱肚子。我们辛勤地耕耘着,就连粪便也奉献给了这片土地,可是我们得到的除去身上的这张皮就一无所有啦。你们,对,就是我面前的奶牛们,你们去年整整一年产的奶呢?那几千加仑的本可以哺育很多健壮的牛犊的奶去哪里了呢?它们一滴不剩地流进了我们仇敌的肚子里了。母鸡们,你们产的蛋呢?那么多的蛋有多少孵了小鸡?没孵化的蛋都去了市场为琼斯和他的伙计们换取了大把的钞票!还有你,可怜的苜蓿,你的那四个孩子呢,那本应是你晚年依靠的孩子去哪了呢?他们早在一岁的时候就被卖掉了。你四次分娩,一贯的勤劳苦干为你换回的只有勉强糊口的饲料和一间马厩,除此就别无所有啦!

“即便已经这样悲惨了,能得善终却也是一种奢望吧。我算是运气好的,我就不抱怨了。我过的12年算是猪合乎自然的生活,我有四百多个孩子了。但是我们终逃不掉那残忍的一刀。不要说我们猪的不到一年的短暂生命最后会在尖声惊叫中从木架上消失,你们牛、鸡、羊也难以逃脱这种恐怖的厄运。当然,马和狗不在例外里。拳师啊,你的肌肉有一天也会失去强健的力量,那时,琼斯就会把你卖给专门屠宰废马的贩子,宰完煮熟了喂猎狐犬。还有狗,琼斯会等到他们老了,没有牙的时候在脖子上绑块砖沉到最近的池塘里。

“那么,你们明白了吗,同志们,我们的苦难就是人类的暴虐统治所造成的,这一点毋庸置疑!只有推翻了人类的统治,我们才会富足和自由,也许就在一夜之间。所以,我要告诉你们的就是:造反!同志们,这个目标一定要牢记,而且要一代一代传下去。我不知道会在什么时候造反,也许一周内,也许百年后,但是正是因为正义一定会得到伸张,这样正如我脚边的干稻草一样坚定不移的道理让我坚信,未来的动物们会前赴后继地完成这个理想。

“要牢记,人是永远不会考虑其他生灵的,他们只会为了自己的利益,我们决不能相信他们的谎言。什么人与动物有着共同的利益,什么人的兴盛就是动物的兴盛,那全是花言巧语。所有的动物,我们是同志,我们要建立纯粹的友谊,我们必须团结一致,与所有的人为敌。”

正在这时,几只狗看见了蹲坐在那儿听演讲的四只大如斗的老鼠,要不是他们及时地窜回洞里,恐怕性命难保,少校举起前蹄平息了这场骚动。

他接着说道:“同志们,这个意外就需要我们弄清楚一点,老鼠和兔子这种非家养的动物应该算在哪一列里?同志,还是仇敌?我们现在就通过大会来表决。”

除了三条狗和一只猫投了反对票,其他的动物都同意把老鼠作为同志。事后发现,猫既投了反对票也投了赞成票。少校继续说道:

“除此之外我只想再强调一下,对待人类及其他们的行为我们必须抱着不共戴天、他们必亡的信念,这是我们的重任。如何划分敌我我来定一下,两条腿行走的就是仇敌,四条腿或者长翅膀的就是我们的同志。我们须谨记一条,我们决不能效仿我们的仇敌,无论是反抗中还是成功后,他们的恶习我们坚决摒弃,因为那些习惯是邪恶的。他们的房间我们决不能住进去,我们不准睡床,不准穿衣喝酒和抽烟,并且从事交易和接触货币也是严令禁止的。哦,我们不能像人类一样残暴地对待我们自己,不论瘦弱还是强壮,不论聪明还是愚蠢,我们是同志,是兄弟,我们之间是平等的,我们要相亲相爱,不能相互屠杀。

“好了,现在该是我们讲讲我昨晚那个梦的时候了。与其说是梦,不如说是一个梦想,一个关于未来的,人类消亡之后的世界的梦想。那是无法用言语描述的一个梦啊。它让我想起了被我遗忘了很久的一些事情。那首歌,那首在我还是小猪时,母亲和其他母猪就会经常哼哼的老歌,虽然她们只会哼个曲调和头三句。那时候对这首老歌我就很熟悉了,即使后来忘记了,但是昨晚的梦让我立刻就想了起来。更妙的是,梦中连歌词都出来了。那是失传了很久的歌词,即使很久很久之前动物们经常唱。我年纪大了,嗓音也不好了,我现在把这首歌唱给你们听,等你们学会了,你们肯定能唱得比我好听,这首歌就叫《英格兰的畜牧》。”

老少校清了清嗓子,但出口的声音仍是很沙哑,即使嗓音不尽如人意,仍是唱得慷慨激昂,旋律有点介于《克莱门汀》和《拉酷酷拉恰》之间。歌词是这样的:

英格兰兽,爱尔兰兽,

普天之下的兽,

倾听我喜悦的佳音,

金色的未来里那可喜的佳音。

这一天终将会到来,

暴虐的人类终将消灭,

富饶的英格兰大地,

将只留下我们的足迹。

我们的鼻中不再扣环,

我们的背上不再配鞍,

橛子和马刺会永远锈蚀

,

不再有残酷的鞭子噼啪作响。

那就连在梦中都难以出现的富裕生活,

小麦、大麦、干草、燕麦

,

苜蓿、大豆还有甜菜,

那一天将全归我所有。

那一天我们将自由解放,

阳光普照英格兰大地,

水会更清纯,

风也更醉人。

哪怕我们活不到那一天,

但为了那一天我们岂能等闲,

牛、马、鹅、鸡

,

为自由必须流血汗。

英格兰兽、爱尔兰兽,

普天之下的兽,

倾听我喜悦的佳音,

金色的未来里那可喜的佳音。

听着这激昂的歌唱,动物们进入了空前的亢奋状态,

他们跟着老少校开始学唱,没一会儿,那些聪明一些的猪狗就把整首歌都学会了,就连最愚笨的动物都学会了些曲调和几句歌词。几番尝试下来,突然间歌声响彻整个庄园,惊人的一致,母牛哞哞叫,狗汪汪地吠,羊咩咩地配合,马的嘶鸣声和鸭子的嘎嘎声也恰到好处,动物们太激动了,以至于连着唱了五遍还没有停下来的意思。

但遗憾的是声音太大了,大到把琼斯先生给吵醒了。他以为是狐狸来光临农舍了,所以他跳下床,拿起放在卧室角落的那杆枪,对黑暗处放了一枪,于是这次聚会就在这枚6号子弹射进大谷仓的墙上后匆匆结束了。顷刻间整个庄园就鸦雀无声了,家畜卧倒在干草堆里,家禽跳上了栖木架。
