\chapter{第三章}

动物们这次真是干劲十足,流了大把的汗水,终于换回了大丰收,这次收获的牧草比他们预计的要多得多。

其实,动物们在干活中遇到了不少难题,干起来也是费力的,因为那些工具是为人设计的,动物们最大的劣势就是不能两条腿站立操作那些工具。不过,猪们总是有办法,即便他们实际上并不干活,只是负责指挥和监督,算起来他们算是学识最出众的,所以理所应当地被所有动物认作了领导。马呢,因为常年在这片土地上劳作,对这里的一尺一寸都是熟悉的,其实说起对于割草和耕地的精通程度,琼斯和他的伙计是比不得的。所以,拳师和苜蓿非常自觉地套上了割草机和马拉耙机(是没有嚼子和缰绳的,因为已经焚毁了),他俩的步伐稳健,坚定地一圈一圈行进,猪跟在他俩后面,一边绕,一边嘴里吆喝着“驾,加油,朋友!”,一会儿又是“吁,停下,同志!”。其他的动物也没闲着,都参与到了搬运和堆积牧草的紧张工作中,就连鸡和鸭子也不辞辛劳地在太阳光的焦灼下用嘴衔起一小撮牧草来回地奔波。他们卖力地干活换回了大丰收,这是庄园从未曾有过的大丰收啊,并且比以前琼斯和他伙计们在的时候整整提前了两天。动物们都很自觉,没有偷吃一口,并且也没有落下一根草秆草叶,这全靠了眼尖的鸡和鸭子。

一整个夏天,庄园里的一切都井井有条,没有丝毫混乱。动物们过上了前所未有的幸福日子,这之前都不敢奢望啊,食物完全属于他们,为了自己的食物所以自己生产,并不是为了曾经主人的施舍,这让他们真真正正体验到当家做主的感觉,以前吃饭是为了活命,而现在的食物带给他们的则是享受。自从那次暴动成功,驱赶了琼斯那些寄生虫般的人,即便动物们没什么经验,但是也有了更多的食物和更多的时间休息。一切仿佛都是那么顺利,即便动物们遇到了很多麻烦,但是一件件地都迎刃而解。比如,下半年收完谷物,就要脱粒,但是庄园里没有脱粒机,于是那种最原始的方法被搬了出来,动物们齐上阵,把谷子揪下来,再用嘴把壳吹掉。对于这些困难出力最多的就是猪和拳师了,猪出的是脑力,拳师出的是体力。拳师一直都是勤劳勤奋的,即便是在琼斯时期也是如此,从不偷懒,现在更是卖力,一个顶仨。哪里工作最艰苦,哪里就有他的身影。有一段时间,基本上庄园里所有活都落在了拳师的肩上,从早到晚,他不停歇地推呀拉呀。为了能多干点活(不在他范围内的活),他早就跟一只小公鸡约定了,每天提前半小时叫醒他,他对所干的活唯一挑剔的就是必须看起来是最累最出力的,他的座右铭就定为了“我要更加努力工作”,这句话在每当遇见挫折和困难的时候就会被他念出来。

但是,每个动物都必须量力而行,比如鸡和鸭,他们单就是捡拾那些遗落的谷粒,就为大家增加了一百八十升的粮食。现在大家很团结,不会有人偷吃,该多少就多少,大家不会为了粮食份额争来争去,再也不见了以往的钩心斗角,争吵打斗。大家都勤勤恳恳,几乎没有动物偷懒,但是例外是不可避免的,莫丽就很不愿意早起,她还常常借故蹄子里夹了石头为由,扔下工作溜走了。猫也表现得与众不同,大家也渐渐发现,每当有活干的时候就不见了她的踪影,直到开饭或者大家收工后她才会若无其事地出现,但是她总有理由为自己辩解,咕噜咕噜说着貌似有理的原因,但偏就是真诚得让大家没办法怀疑她良好的动机。那头老驴本杰明对待这次造反的态度很不明朗,因为他在起义前后没有什么太大的变化,他一直是慢条斯理,本本分分干着属于自己的活,既不多干,也不少干,但从不旷工。也有多事的动物问本杰明是不是比以前好,但他没有正面回答,只是说:“驴子都是长寿的动物,你们谁见过死驴吗?”对于这神秘的答案,动物们也就作罢了。

星期天主要是休息的,不用干活,早餐也比平时晚一个小时,吃过早餐,就要完成一系列的仪式,而且每周如此。首先就是升旗,旗子是用琼斯夫人的旧的绿桌布做成的,是雪球从农具室翻出来的,他在上面用白漆画上了一只蹄子和一只犄角,之后这面旗就会在每个星期天早上在庄主院子内花园的旗杆上升起。对于旗子的含义,雪球是这样讲解的,绿色代表了整个英格兰田野,白色的蹄子和犄角象征了未来的动物共和国,这个共和国会在人类被彻底铲除后诞生。升旗之后,所有动物排着队依次进入大谷仓,参加名为碰头会的全体大会。会议的内容主要就是下一周的工作安排和关于各项提议的讨论和结果。猪是提案的主力军,其他的动物只有表决的份,但他们对这毫不介意。拿破仑和雪球总是最显眼的,但大家都瞧得出来,他俩不合,只要其中一个赞成什么,另外一个就反对什么。即便是已经通过的议案,他俩也会揪住其中的问题而争执不休,比如之前有个议案,是把果园后面的一小片牧场留给过了劳动年龄的动物们作为养老地,大家对这个方案都没有反对,但是在退休年龄上,双方又各执己见,激烈不已啊。会议的最后就是演唱那首《英格兰兽》。下午的时间留给大家自行娱乐。

农具室被猪指定为他们的指挥部,他们把里面的农具倒腾到了其他地方。每到晚上,他们就会来到这里,要么学习打铁,要么学习木工,或者其他生活中经常能用到的技艺,当然他们一定要参照琼斯留下来的那些书本。雪球一直在忙着组织他的“动物委员会”机构,对此他尽心尽力,尽管累得半死,但是依然乐此不疲。他之前为动物做过很多这样的事,但几乎都不幸地失败了,比如属于母鸡的“产蛋委员会”,属于牛的“清洁尾巴社”,有为羊发起的“羊毛增白运动”,还有专为驯化老鼠和野兔的“野生同志再教育委员会”。就如那个驯化野生动物的努力很快就被宣布是白费的,那些个野生动物的行为几乎没什么变化,要是对他们不计较的话,他们就明目张胆地钻空子。猫也参加了这个委员会,她倒是在这里找到了乐趣。有动物看见她曾经有一天站在窝棚顶与她够不着的一些麻雀聊天。她对麻雀们说,现在动物们都是同志了,所以,只要麻雀们愿意,就可以站到她的爪子上来,但是麻雀们却照样对她能离多远就离多远。

但是,有一个机构却带来了很大的效益,那就是读写班。到了秋风吹起的时候,庄园里的动物们都不同程度地有了点文化,至少不是文盲了。

读和写对于猪这种聪明的动物来说太容易了。狗对阅读掌握得也不错,但是他们只对“七诫”有兴趣。山羊穆尔丽跟狗比起来还略胜一筹,因为她时常在傍晚从垃圾堆里翻出报纸碎片念给其他动物听。驴子本杰明的阅读水平甚至可以和猪匹敌,但是他不以为意,他说还没有什么是值得阅读的。苜蓿学会了二十六个字母,但是却不会拼成单词。拳师只能记住A、B、C、D,他会用大蹄子在沙地上写完这四个字母,然后就会目不转睛,偶尔抖一下额毛,冥思苦想下一个字母,却怎么都想不起来。当然他也学过E、F、G、H,但是会了这几个,A、B、C、D却忘了。于是他最后决定就只记住前四个吧,为了能记得更牢,他每天都会写上一两遍。莫丽只会那五个能拼出她名字的字母Mollie,不是智商不够,是她根本不学其他的。她会用几根细嫩的树枝,很灵巧地把她的名字拼出来,然后再用几朵鲜花稍加装饰,再绕着走几圈,不住地赞叹。

庄园里的其他动物大都不能越过A字母。有些智商很低的动物就连“七诫”也记不住,比如羊、鸡和鸭子。鉴于这种情况,雪球经过反复研究,为这些动物量身定做了一句话——“四条腿好,两条腿坏”,这句话是从“七诫”里提炼出来的,它涵盖了兽主义的精髓,一旦掌握了它,人类的威胁就会被免除了。但是鸟类、家禽类就不愿意了,因为两条腿不好,他们只有两条腿,对于这,雪球给了他们一个看似合理的解释。

雪球说:“禽鸟们,你觉得你们只有两条腿吗?你觉得你们和人类没有区别吗?这种想法是错误的。你们的翅膀,是推进行进的器官,并不是用来操控的,所以它们的功能就是腿。而人类的手是作恶多端的器官。”

禽鸟们对于这番解释并不是很明白,但是他们还是接受了。于是,所有智商都不太够用的动物都一丝不苟地去记这个简单的新准则。这条新准则也被写在大谷仓的墙上,“七诫”的上面,甚至字要更大一些。绵羊们对这句短小精悍的准则无比上心,很快他们就记住了,以后每当他们躺在地上,就会连续不断地重复着“四条腿好,两条腿坏”,可以一叫就是几个小时,对此,他们从不觉得厌烦。

对于雪球做的这些建立委员会的工作,拿破仑不屑一顾。他觉得目前最重要的事情应该是对年轻一代的教育问题。赶巧的是收割牧草之后不久,杰西和蓝铃铛一共产下了九只健壮的小狗仔,这些小狗断奶的那天,拿破仑就把他们带走了,说是为了他们的教育,把他们带到了一间仅能用曾经放在农具室里的那架梯子爬上去的阁楼,方便他们与外界隔离,不出意外,庄园里的动物们很快就遗忘了这九只小狗的存在。

之前消失了的牛奶,原来是放进了猪每天的饲料里。果园里有很多早茬的苹果被吹落在了草坪上,动物们原本认为会分给大家享用,但是一条新指令击碎了他们天真的想法。指令要求凡是被吹落的苹果一定要收集起来,储藏在农具室里,供猪们食用。动物们对此很是不满,他们嘟嘟囔囔地抱怨,但是无济于事,奇怪的是,那两头凡事都是死对头的猪在这件事上竟然没有争执。声响器马上发挥他的特长,去向那些动物作些必要的解释。

“同志们,”声响器大声地叫道,“你们会觉得我们猪这么做是很自私,滥用权力吗?其实不是的,完全不是。你们千万不要这么想。其实啊,我们中有很多猪都是不喜欢吃苹果和牛奶的,比如我就很不喜欢。但是我们为什么要食用呢?那其实是为了保证我们的身体健康。科学研究表明,牛奶和苹果中含有对猪来说必不可少的营养物质。你们想啊,我们猪是从事脑力劳动的,庄园里的所有工作都需要我们去管理,去组织,我们没日没夜地为大家操劳,所以,为了不因缺乏营养而生病,为了不因生病而耽误工作,我们猪才去吃那些苹果,喝那些牛奶的。我们都是为了大家好,你们想,如果我们病倒了,将会发生多么严重的事啊,那个时候,琼斯一定会卷土重来的!一定会的!太可怕了!这是真的,同志们!”声响器用着他百试不厌的绝招,一边蹦来蹦去,一边摇头晃脑,用恳求的语气冲着动物们大声喊道,“我说的都是真的,想必你们也不愿意看到琼斯回来吧?”

一提到琼斯可能会因为这个卷土重来,动物们都不再反对了,因为大家实在是不想让那个事情真的发生,经过声响器这么一说,大家也就觉得让猪们保持健康是抵抗琼斯的必要条件。因此,大家一致同意了:牛奶和吹落的苹果(还包括成熟苹果中的绝大部分)应当作为猪们独享的食物。
